%!TEX root = ../../main.tex
\chapter{Moti vincolati}
\section{Introduzione}

\begin{defn}{Moto soggetto ad un vincolo}{motoSoggettoVincolo}\index{Moto vincolato}
	Diremo che il moto \(\vec{x}\), sotto l'effetto di \(\vec{F}\), \emph{è soggetto ad un vincolo \(\Sigma\)}, se partendo su \(\Sigma\) il moto generato da \(\vec{F}\) su \(\Sigma\) vi rimane per tutti i tempi.
\end{defn}

\begin{ese}
	Abbiamo già visto un esempio di moto vincolato con il pendolo in cui si aveva
	\[
		\vec{x}\in \Sigma \subseteq \R^2 \qquad\text{dove }\Sigma=\Set{\vec{x}:\abs{\vec{x}}=l}.
	\]
\end{ese}
\noindent
Nel contesto dei moti vincolati è conveniente studiare il sistema con coordinate adattate al vincolo.
Supponiamo che \(\Sigma\) sia descritta in forma parametrica
\[
	\vec{x} = \vec{\j}(q_1,\ldots,q_s),
\]
dove \(s\) è la dimensione di \(\Sigma\)

\begin{ese}
	Nel caso del pendolo avevamo
	\[
		\vec{x} = l\, \begin{pmatrix}\sin\q\\-\cos\q\end{pmatrix} \equiv \j(\q).
	\]
\end{ese}
\noindent
Sappiamo che moti reali rendono stazionaria l'azione \(A_{t_1,t_2}^{\mathcal{L}}\).
Sappiamo inoltre che moti che partono su \(\Sigma\) si mantengono su \(\Sigma\). Pertanto moti reali su \(\Sigma\) renderanno, a maggior ragione, stazionaria l'azione di \(A_{t_1,t_2}^{\mathcal{L}}\) rispetto all'azione sull'insieme ristretto dei moti
\[
	\mathcal{M}_{t_1,t_2}^{\Sigma}\big(\vec{\x}_1,\vec{\x}_2\big) \qquad\text{con }\vec{\x}_1,\vec{\x}_2 \in \Sigma;
\]
che è il sottoinsieme di \(M\) tale che \(\vec{x}(t)\in\Sigma\) per ogni \(t\).
D'altronde
\[
	\vec{x}(t) \in \mathcal{M}_{t_1,t_2}^{\Sigma}\big(\vec{\x}_1,\vec{\x}_2\big) \implies \vec{x}(t) = \vec{\j}\big(q_1(t),\ldots,q_s(t)\big)
\]
per cui su tali moti avremo
\[
	\mathcal{L}\big(\vec{x}(t),\dot{\vec{x}}(t)\big) = \tilde{\mathcal{L}}\big(\vec{q}(t),\dot{\vec{q}}(t)\big),
\]
dove \(\tilde{\mathcal{L}}\) è definita come \hyperref[mk:cambiamentoVariabileEquazioneEL]{descritta nel capitolo precedente}.
In questo caso \(s<n\) per cui la lagrangiana \(\tilde{\mathcal{L}}\) sarà ridotta rispetto a \(\mathcal{L}\).

\begin{ese}
	Considerando sempre il pendolo abbiamo
	\[
		\vec{x}(t) = l\,\begin{pmatrix}\sin\q(t)\\-\cos\q(t)\end{pmatrix} \implies \dot{\vec{x}}(t) = l\,\dot{\q}(t)\,\begin{pmatrix}\cos\q(t)\\\sin\q(t)\end{pmatrix}.
	\]
	Inoltre \(U=m\,g\,x_2\), per cui
	\[
		\tilde{\mathcal{L}}\big(\q(t),\dot{\q}(t)\big) = \frac{1}{2}m\,l^2\,\dot{\q}^2 + m\,g\,\cos\q.
	\]
	Possiamo quindi applicare le equazioni di Eulero-Lagrange
	\[
		\frac{\dd}{\dd t}\frac{\pd\tilde{\mathcal{L}}}{\pd\dot{\q}} = \frac{\pd\tilde{\mathcal{L}}}{\pd\q} \iff \frac{\dd}{\dd t} m\,l^2\,\dot{\q}(t) = -m\,g\,l\,\sin\big(\q(t)\big),
	\]
	da cui otteniamo l'equazione ridotta del moto
	\[
		\ddot{\q} = -\frac{g}{l}\sin\q.
	\]
\end{ese}
\noindent
Osserviamo che il procedimento applicato nell'esempio precedente al pendolo, ha avuto successo nonostante fossimo in presenza di forze non conservative. In generale infatti, un sistema di forze \(\vec{F}\) che mantiene \(\vec{x}\) sul vincolo non è conservativo e quanto detto non si applica.
D'altronde è possibile modificare opportunamente la trattazione precedente ed assumere che la forza di vincolo sia conservativa.

Nel caso del pendolo si aveva \(\vec{F}=m\,\vec{g}+\vec{T}\), dove
\[
	\vec{T} = -\hat{e}_r(m\,g\,\cos\q+m\,l\dot{\q}^2)
\]
che è chiaramente non conservativa (ad esempio non è posizionale).
Osserviamo che \(\vec{F}\) si decompone naturalmente in una componente conservativa e in una non conservativa ortogonale al vincolo.
In particolare segue che \(\vec{T}\) non compie lavoro sui moti su \(\Sigma\). Questo ci permetterà di dimostrare che tali moti sono trascurabili nella nostra trattazione.

\begin{defn}{Sistema di vincoli}{sistemaVincoli}\index{Vincoli}
	Nella nostra trattazione avremo che, data una superficie di vincolo \(\Sigma\), si ha
	\[
		\vec{x}\in\Sigma \iff 	\begin{cases}
			G_1\big(\vec{x}\big) = 0 \\
			\vdots                   \\
			G_m\big(\vec{x}\big) = 0
		\end{cases}.
	\]
	Dove se \(s\) è la dimensione di \(\Sigma\), si ha \(s=n-m\).
\end{defn}

\begin{notz}
	Vincoli che rispettano tali condizioni si definiscono \emph{olonomi bilateri}.
	In particolare un vincolo olonomo è indipendente dalle velocità; mentre un vincolo bilatero è descritto da una condizione del tipo \(G\big(\vec{x}\big)=0\).
\end{notz}
%%%%%%%%%%%%%%%%%%%%%%%%%%%%%%%%%%%%%%%%%%
%
%LEZIONE 24/04/2017 - OTTAVA SETTIMANA (1)
%
%%%%%%%%%%%%%%%%%%%%%%%%%%%%%%%%%%%%%%%%%%
\section{Vincoli ideali}

\begin{defn}{Sistema di vincoli regolari}{sistemaVincoliRegolari}\index{Vincoli!regolari}
	Un sistema di vincoli
	\[
		\begin{cases}
			G_1\big(\vec{x}\big) = 0 \\
			\vdots                   \\
			G_m\big(\vec{x}\big) = 0
		\end{cases}
		\qquad\text{con }\vec{x}\in\R^n, G_i\in C^{\infty}(\R^n,\R),
	\]
	si dice \emph{regolare} se è tale che
	\[
		\frac{\pd G_1}{\pd\vec{x}}\big(\vec{x}\big), \ldots, \frac{\pd G_m}{\pd\vec{x}}\big(\vec{x}\big),
	\]
	sono \(m\) vettori indipendenti per ogni \(\vec{x}\) che soddisfa il sistema.
\end{defn}

\begin{oss}
	Un vincolo regolare definito da \(m\) equazioni ha la superficie \(\Sigma\) di dimensione \(s=n-m\).
\end{oss}

\begin{ese}
	Nel caso del pendolo abbiamo visto che \(G\big(\vec{x}\big) = \abs{\vec{x}}^2-l^2\), ora
	\[
		\frac{\pd G}{\pd\vec{x}}\big(\vec{x}\big) = 2\vec{x} \neq 0,\,\fa \abs{\vec{x}} = l,
	\]
	per cui il vincolo è regolare.
\end{ese}

\begin{teor}{Sistema locale di coordinate adattate al vincolo}{sistemaLocaleCoordinateVincolo}
	Sia \(\Sigma\) una superficie di vincolo associata ad un sistema di vincoli regolari.
	Allora \(\Sigma\) ammette, almeno localmente attorno ad ogni \(\vec{x}_0\in\Sigma\), un sistema di coordinate adattato al vincolo.
	Ovvero, se \(U_0\) è un aperto di \(\R^n\) e \(U_1\) un intorno di \(\vec{x}_0\in\Sigma\), esiste \(\vec{X}\colon U_0 \longleftrightarrow U_1\) tale che
	\[
		\vec{x} = \vec{X}(q_1,\ldots,q_s,q_{s+1},\ldots,q_n) \in \Sigma\cap U_1 \iff q_{s+1} = \ldots = q_n = 0.
	\]
\end{teor}

\begin{proof}
	Segue dal teorema della funzione implicita.
\end{proof}

\begin{oss}
	La mappa
	\[
		\vec{\j}(q_1,\ldots,q_s) \equiv \vec{X}(q_1,\ldots,q_s,0,\ldots,0),
	\]
	fornisce una parametrizzazione locale di \(\Sigma\) in un intorno di \(\vec{x}_0\).
\end{oss}

\begin{ese}
	Nel caso del pendolo \(G\big(\vec{x}\big)=\abs{\vec{x}}^2-l^2\).
	Un sistema di coordinate adattate è fornito dalla seguente mappa
	\[
		\vec{x} = \vec{X}(\q,\d) = (l+\d)\,\begin{pmatrix}\sin\q\\-\cos\q\end{pmatrix},
	\]
	posto \(\d=0\) troviamo
	\[
		\vec{x} = l\,\begin{pmatrix}\sin\q\\-\cos\q\end{pmatrix} \equiv \vec{\j}(\q)
	\]
	che è una parametrizzazione, in questo caso specifico globale, del vincolo.
\end{ese}

\begin{defn}{Velocità compatibile con il vincolo}{velocitàCompatibileVincolo}
	Diremo che \(\dot{\vec{x}}(t_0)\) è \emph{compatibile} con il sistema di vincoli \((G_1,\ldots,G_m)\) se è tangente alla superficie di vincolo.
\end{defn}

\begin{oss}
	Tale definizione è naturale, infatti se \(\vec{x}(t)\in\Sigma\) avremo
	\[
		\begin{cases}
			G_1\big(\vec{x}(t)\big) = 0 \\
			\vdots                      \\
			G_m\big(\vec{x}(t)\big) = 0
		\end{cases} \implies
		\begin{cases}
			\dot{\vec{x}}(t) \sdot \frac{\pd G_1}{\pd\vec{x}}\big(\vec{x}(t)\big) = 0 \\
			\vdots                                                                    \\
			\dot{\vec{x}}(t) \sdot \frac{\pd G_m}{\pd\vec{x}}\big(\vec{x}(t)\big) = 0
		\end{cases} \iff \dot{\vec{x}} \perp \frac{\pd G_i}{\pd\vec{x}}\big(\vec{x}\big)\,\fa i,
	\]
	ovvero \(\dot{\vec{x}} \in T_{\vec{x}}\Sigma\), dove \(T_{\vec{x}}\Sigma\) è lo spazio tangente a \(\Sigma\) passante per \(\vec{x}\).
\end{oss}

\begin{defn}{Sistema di punti materiale sottoposto a vincoli regolari}{sistemaPuntiMaterialiVincoliRegolari}
	Consideriamo un sistema di punti materiali descritto dal vettore \(\vec{x}\in\R^n\) e dalla matrice delle masse \(M\), soggetto al sistema di forze \(\vec{F}\).
	Diremo che tale \emph{sistema è sottoposto al sistema di vincoli regolari} \((G_1,\ldots,G_m)\) se ogni dato iniziale \(\vec{x}_0 = \vec{x}(t_0)\in\Sigma\) e \(\vec{v}_0=\dot{\vec{x}}(t_0)\in T_{\vec{x}\Sigma}\), evolve lungo le soluzioni di
	\[
		M\,\ddot{\vec{x}} = \vec{F}
	\]
	in modo tale da rimanere su \(\Sigma\) per ogni tempo.
\end{defn}

\begin{oss}
	Avevamo già osservato che, in generale, un sistema meccanico sottoposto a vincoli regolari è associato ad un sistema di forze \(\vec{F}\) non conservativo, spesso neppure posizionale.
	Infatti tipicamente si ha
	\[
		\vec{F} = \vec{F}\big(\vec{x},\dot{\vec{x}}\big).
	\]
	Detto ciò, in molti casi si osserva che \(\vec{F}\) ha una struttura speciale, ovvero è decomponibile in una forza conservativa e in una non conservativa ortogonale al vincolo.
\end{oss}

\begin{defn}{Forza non conservativa decomponibile}{forzaNonConservativaDecomponibile}
	In un sistema sottoposto a vincoli regolari, consideriamo una forza \(\vec{F}=\vec{F}\big(\vec{x},\dot{\vec{x}}\big)\) non conservativa.
	Diremo che \(\vec{F}\) è \emph{decomponibile} se può essere espressa come somma di una \emph{forza attiva conservativa} e di una \emph{reazione vincolare}, ovvero se
	\[
		\vec{F}\big(\vec{x},\dot{\vec{x}}\big) = -\frac{\pd}{\pd\vec{x}}U\big(\vec{x}\big) + \vec{R}\big(\vec{x},\dot{\vec{x}}\big).
	\]
\end{defn}

\begin{defn}{Principio di D'Alembert}{principioDAlembert}\index{Principio di D'Alembert}
	Consideriamo \(\vec{R}\) la reazione vincolare di una forza decomponibile.
	Diremo che \(\vec{R}\) soddisfa il \emph{principio di D'Alembert} se \(\vec{R}\) è ortogonale al vincolo \(\Sigma\). Ovvero se
	\[
		\vec{R}\big(\vec{x},\vec{v}\big) \sdot \vec{\h} = 0 \qquad\text{per ogni }\vec{v},\vec{\h} \in T_{\vec{x}}\Sigma.
	\]
\end{defn}

\begin{oss}
	Possiamo pensare alla \(\vec{\h}\) nella definizione come allo spostamento infinitesimo da \(\vec{x}\) a \(\vec{x}+\vec{\h}\,\dd t\) associato ad una velocità \(\vec{\h}\).
\end{oss}

\begin{defn}{Sistema di vincoli ideali}{sistemaVincoliIdeali}\index{Vincoli!ideali}
	Consideriamo un sistema meccanico \(M\,\ddot{\vec{x}}=\vec{F}\big(\vec{x},\dot{\vec{x}}\big)\) sottoposto al sistema di vincoli regolari \((G_1,\ldots,G_m)\).
	Diremo che il sistema di vincoli è \emph{ideale} rispetto alla decomposizione
	\[
		\vec{F}\big(\vec{x},\dot{\vec{x}}\big) = -\frac{\pd}{\pd\vec{x}}U\big(\vec{x}\big) + \vec{R}\big(\vec{x},\dot{\vec{x}}\big)
	\]
	se \(\vec{R}\) soddisfa il principio di D'Alembert.
\end{defn}

\section{Sistemi di punti materiali sottoposti a sistemi di vincoli ideali}

\begin{teor}{Stazionarietà per sistemi di vincoli ideali}{stazionarietàSistemiVincoliIdeale}
	Consideriamo un sistema meccanico \(M\,\ddot{\vec{x}}=\vec{F} =-\frac{\pd U}{\pd\vec{x}}+\vec{R}\) sottoposto al sistema di vincoli ideali \((G_1,\ldots,G_m)\).
	I moti reali compatibili con il vincolo rendono stazionaria l'azione di Lagrangiana
	\[
		\mathcal{L}\big(\vec{x},\dot{\vec{x}}\big) = \frac{1}{2}\dot{\vec{x}} \sdot M\,\dot{\vec{x}} - U\big(\vec{x}\big)
	\]
	rispetto a variazioni in \(\mathcal{V}_{\vec{x}}\Big(\mathcal{M}_{t_1,t_2}^{G_1,\ldots,G_m}\big(\vec{\x}_1,\vec{\x}_2\big)\Big)\).
\end{teor}

\begin{proof}
	I moti reali compatibili con il vincolo sono le \(\vec{x}(t)\) soluzioni del sistema tali che \(\vec{x}(t)\in\Sigma\,\fa t\). Vogliamo quindi mostrare che tale \(\vec{x}(t)\) rende stazionaria l'azione
	\[
		A\big[\vec{x}(t)\big] = \int_{t_1}^{t_2}\left[\frac{1}{2}\dot{\vec{x}}(t)\sdot M\,\dot{\vec{x}}(t)-U\big(\vec{x}(t)\big)\right]\,\dd t,
	\]
	rispetto a variazioni \(\vec{y}_\e\in\mathcal{V}_{\vec{x}}\Big(\mathcal{M}_{t_1,t_2}^\Sigma\big(\vec{\x}_1,\vec{\x}_2\big)\Big)\) con \(\vec{\x}_1,\vec{\x}_2\in\Sigma\).
	Ricordiamo inoltre che \(\vec{y}_\e(t)\) è tale che
	\[
		\vec{y}_0(t) = \vec{x}(t), \qquad \vec{y}_\e(t_1) = \vec{\x}_1, \qquad \vec{y}_\e(t_2) = \vec{\x}_2, \qquad \vec{y}_\e(t)\in\Sigma\,\fa t.
	\]
	Affinché l'azione sia resa stazionaria si deve avere
	\[
		\frac{\dd}{\dd\e}A\big[\vec{y}_\e(t)\big]\bigg|_{\e=0} = 0.
	\]
	Ora
	\[
		\vec{y}_\e(t) = \vec{x}(t)+\e\,\vec{z}(t)+\bO(\e^2) \qquad\text{con }\vec{z}(t) = \frac{\pd}{\pd\e}\vec{y}_\e(t)\bigg|_{\e=0}.
	\]
	Per cui
	\[
		\begin{split}
			\frac{\dd}{\dd\e}A\big[\vec{y}_\e(t)\big]\bigg|_{\e=0} & = \frac{\dd}{\dd\e}\int_{t_1}^{t_2}\left[\frac{1}{2}\big(\dot{\vec{x}}(t)+\e\,\dot{\vec{z}}(t)\big)\sdot M\,\big(\dot{\vec{x}}(t)+\e\,\dot{\vec{z}}(t)\big)-U\big(\vec{x}(t)+\e\,\vec{z}(t)\big)\right]\,\dd t \bigg|_{\e=0}\\
			& = \int_{t_1}^{t_2}\Big[\dot{\vec{x}}(t)\sdot M\,\dot{\vec{z}}(t)-\frac{\pd U}{\pd\vec{x}}\big(\vec{x}(t)\big)\sdot \vec{z}(t)\Big]\,\dd t.
		\end{split}
	\]
	Ricordiamo che \(\vec{z}(t_1)=\vec{z}(t_2)=\vec{0}\) e integriamo per parti il primo addendo dell'integranda:
	\[
		\frac{\dd}{\dd\e}A\big[\vec{y}_\e(t)\big]\bigg|_{\e=0} = \int_{t_1}^{t_2}\Big[-M\,\ddot{\vec{x}}(t)-\frac{\pd U}{\pd\vec{x}}\big(\vec{x}(t)\big)\Big]\sdot \vec{z}(t)\,\dd t = - \int_{t_1}^{t_2}\Big[M\,\ddot{\vec{x}}(t)+\frac{\pd U}{\pd\vec{x}}\big(\vec{x}(t)\big)\Big]\sdot \vec{z}(t)\,\dd t
	\]
	Sfruttiamo l'equazione del moto
	\[
		M\,\ddot{\vec{x}} = -\frac{\pd U}{\pd\vec{x}}\big(\vec{x}\big) + \vec{R}\big(\vec{x},\dot{\vec{x}}\big) \implies M\,\ddot{\vec{x}} + \frac{\pd U}{\pd\vec{x}}\big(\vec{x}\big) = \vec{R}\big(\vec{x},\dot{\vec{x}}\big),
	\]
	da cui
	\[
		\frac{\dd}{\dd\e}A\big[\vec{y}_\e(t)\big]\bigg|_{\e=0} = -\int_{t_1}^{t_2}\vec{R}\big(\vec{x}(t),\dot{\vec{x}}(t)\big)\sdot \vec{z}(t)\,\dd t = 0,
	\]
	dove l'integranda è nulla per il principio di D'Alembert.
	Il principio è verificato a patto che \(\dot{\vec{x}},\vec{z}\in T_{\vec{x}}\Sigma\).
	Questo è banalmente vero per \(\dot{\vec{x}}\) poiché i dati iniziali sono compatibili con il vincolo, mostriamolo quindi per \(\vec{z}\). Di fatto osserviamo che \(\vec{z}(t)\in T_{\vec{x}}\Sigma\) per costruzione; ricordiamo infatti che
	\[
		\vec{y}(\e,t) = \vec{y}_\e(t)\in\Sigma,\,\fa t \in [t_1,t_2].
	\]
	Consideriamo quindi il \emph{moto virtuale}
	\[
		\vec{\tilde{x}}(s) = \vec{y}(s-t,t),
	\]
	con \(t\) fissato e \(s\in(t-\d\,t,t+\d\,t)\) che appartiene al vincolo per ogni \(s\) ed ha velocità
	\[
		\vec{\h} = \frac{\dd}{\dd s}\vec{\tilde{x}}(s)\bigg|_{s=t} = \frac{\dd}{\dd\e}\vec{y}_\e(t)\bigg|_{\e=0} = \vec{z}(t).\qedhere
	\]
\end{proof}

\begin{notz}
	Con \(\mathcal{M}_{t_1,t_2}^{G_1,\ldots,G_m}\big(\vec{\x}_1,\vec{\x}_2\big)\) indichiamo 
	\[
		\mathcal{M}_{t_1,t_2}^\Sigma = \Set{\vec{x}(t) \in \mathcal{M}_{t_1,t_2}\big(\vec{\x}_1,\vec{\x}_2\big) | \vec{x}(t) \in \Sigma,\,\fa t \in [t_1,t_2]}.
	\]
\end{notz}

\begin{oss}
	Grazie a questo teorema, le equazioni del moto sul vincolo in coordinate adattate \(\vec{q}\) sono semplicemente le equazioni di Eulero-Lagrange associate  alla lagrangiana ridotta
	\[
		\tilde{\mathcal{L}}\big(\vec{q},\dot{\vec{q}}\big) = \mathcal{L}\big(\vec{x},\dot{\vec{x}}\big)\bigg|_{\substack{\vec{x}=\vec{\j}(\vec{q})\\\dot{\vec{x}}=J(\vec{q})\,\dot{\vec{q}}}} = \frac{1}{2}\Big[J\big(\vec{q}\big)\,\dot{\vec{q}}\Big] \sdot M\,\Big[J\big(\vec{q}\big)\,\dot{\vec{q}}\Big] - U\big(\vec{\j}(\vec{q})\big).
	\]
	dove \(\vec{q}=(q_1,\ldots,q_s)\) con \(s=n-m\) e \(\vec{x}=\vec{X}(q_1,\ldots,q_s,0,\ldots,0)=\vec{\j}(q_1,\ldots,q_s)\). Ovvero per
	\[
		\tilde{\mathcal{L}}\big(\vec{q},\dot{\vec{q}}\big) = \frac{1}{2}\dot{\vec{q}} \sdot \tilde{M}\big(\vec{q}\big)\,\dot{\vec{q}} - V\big(\vec{q}\big),
	\]
	dove
	\[
		\tilde{M}\big(\vec{q}\big) = \tran{J}\big(\vec{q}\big)\,M\,J\big(\vec{q}\big) \qquad\text{e}\qquad V\big(\vec{q}\big) = U\big(\vec{\j}(\vec{q})\big).
	\]
\end{oss}
%%%%%%%%%%%%%%%%%%%%%%%%%%%%%%%%%%%%%%%%%%
%
%LEZIONE 26/04/2017 - OTTAVA SETTIMANA (2)
%
%%%%%%%%%%%%%%%%%%%%%%%%%%%%%%%%%%%%%%%%%%
\section{Considerazione per casi dipendenti dal tempo}

In questo paragrafo rivedremo alcune definizioni e teoremi esposti in precedenza nel caso in cui la dipendenza dal tempo sia esplicita. La dimostrazione dei teoremi verrà tralasciata in quanto analoga al caso stazionario.

Per quanto riguarda la notazione, molte definizioni avranno lo stesso nome; è quindi lasciata sottintesa la dipendenza dal tempo nella nomenclatura.

\begin{defn}{Sistema di vincoli}{sistemaVincoliTempo}
	Consideriamo una superficie di vincolo \(\Sigma(t)\). Diremo che \(\vec{x}\in\Sigma(t)\) se e soltanto se
	\[
		\vec{x} \text{ soddisfa } 	\begin{cases}
			G_1\big(\vec{x},t\big) = 0 \\
			\vdots                     \\
			G_m\big(\vec{x},t\big) = 0
		\end{cases}
	\]
\end{defn}

\begin{defn}{Sistema di vincoli regolare}{sistemaVincoliRegolareTempo}
	Un sistema di vincoli \((G_1,\ldots,G_m)\) si dice \emph{regolare} se gli \(m\) vettori
	\[
		\frac{\pd G_1}{\pd\vec{x}}\big(\vec{x},t\big), \ldots, \frac{\pd G_m}{\pd\vec{x}}\big(\vec{x},t\big),
	\]
	sono indipendenti per ogni \(\vec{x}\in\Sigma(t)\).
\end{defn}
\noindent
Osserviamo che se \(\vec{x}(t)\in\Sigma(t)\), allora
\[
	\begin{cases}
		\dot{\vec{x}}(t) \sdot \frac{\pd G_1}{\pd\vec{x}}\big(\vec{x}(t),t\big) + \frac{\pd G_1}{\pd t}\big(\vec{x}(t),t\big) = 0 \\
		\vdots                                                                                                                    \\
		\dot{\vec{x}}(t) \sdot \frac{\pd G_m}{\pd\vec{x}}\big(\vec{x}(t),t\big) + \frac{\pd G_m}{\pd t}\big(\vec{x}(t),t\big) = 0
	\end{cases}
\]
Da ciò segue una prima differenza con il caso indipendente dal tempo. Dato \(\vec{x}_0\in\Sigma(t)\) diremo infatti che \(\vec{v}_0\) è una velocità compatibile con il sistema di vincoli associati al punto \(\vec{x}_0\) se
\[
	\begin{cases}
		\vec{v}_0 \sdot \frac{\pd G_1}{\pd\vec{x}}\big(\vec{x}_0,t_0\big) + \frac{\pd G_1}{\pd t}\big(\vec{x}_0,t_0\big) = 0 \\
		\vdots                                                                                                               \\
		\vec{v}_0 \sdot \frac{\pd G_m}{\pd\vec{x}}\big(\vec{x}_0,t_0\big) + \frac{\pd G_m}{\pd t}\big(\vec{x}_0,t_0\big) = 0
	\end{cases}
\]
ma ciò non equivale, come nel caso stazionario, all'appartenere allo spazio tangente del vincolo. Infatti
\[
	\vec{\h} \in T_{\vec{x}_0}\Sigma(t_0) \iff 	\begin{cases}
		\vec{\h} \sdot \frac{\pd G_1}{\pd \vec{x}}\big(\vec{x}_0,t_0\big) = 0 \\
		\vdots                                                                \\
		\vec{\h} \sdot \frac{\pd G_m}{\pd \vec{x}}\big(\vec{x}_0,t_0\big) = 0
	\end{cases}
\]

\begin{teor}{Sistema locale di coordinate adattate al vincolo}{sistemaLocaleCoordinateVincoloTempo}
	Sia \(\Sigma(t)\) una superficie di vincolo associata ad un sistema di vincoli regolari.
	Allora \(\Sigma(t)\) ammette, almeno localmente attorno ad ogni \(\vec{x}_0\in\Sigma(t_0)\), un sistema di coordinate adattato al vincolo.
	Ovvero, se \(U_0\) è un aperto di \(\R^n\) e \(U_1\) un intorno di \(\vec{x}_0\in\Sigma(t_0)\), esiste \(\vec{X}\colon U_0 \longleftrightarrow U_1\) tale che
	\[
		\vec{x} = \vec{X}(q_1,\ldots,q_s,q_{s+1},\ldots,q_n,t) \in \Sigma(t_0)\cap U_1 \iff q_{s+1} = \ldots = q_n = 0.
	\]
\end{teor}

\begin{oss}
	La mappa
	\[
		\vec{\j}(q_1,\ldots,q_s,t) \equiv \vec{X}(q_1,\ldots,q_s,0,\ldots,0,t),
	\]
	fornisce una parametrizzazione locale di \(\Sigma(t_0)\) in un intorno di \(\vec{x}_0\).
\end{oss}

\begin{defn}{Sistema di vincoli ideali}{sistemaVincoliIdealiTempo}
	Consideriamo un sistema meccanico \(M\,\ddot{\vec{x}}=\vec{F}\big(\vec{x},\dot{\vec{x}},t\big)\) sottoposto al sistema di vincoli regolari \((G_1,\ldots,G_m)\) associati alla superficie di vincolo \(\Sigma(t)\).
	Diremo che il sistema di vincoli è \emph{ideale} rispetto alla decomposizione
	\[
		\vec{F}\big(\vec{x},\dot{\vec{x}},t\big) = -\frac{\pd}{\pd\vec{x}}U\big(\vec{x}\big) + \vec{R}\big(\vec{x},\dot{\vec{x}},t\big)
	\]
	se \(\vec{R}\) soddisfa il principio di D'Alembert. Ovvero
	\[
		\vec{R}\big(\vec{x},\vec{v},t) \sdot \vec{\h} = 0
	\]
	per ogni \(\vec{v}\) compatibile con i vincoli e per ogni \(\vec{\h}\in T_{\vec{x}}\Sigma(t)\).
\end{defn}

\begin{teor}{Stazionarietà per sistemi di vincoli ideali}{stazionarietàSistemiVincoliIdealeTempo}
	Consideriamo un sistema meccanico \(M\,\ddot{\vec{x}}=\vec{F}\big(\vec{x},\dot{\vec{x}},t\big)\) sottoposto al sistema di vincoli ideali \((G_1,\ldots,G_m)\) associato alla superficie di vincolo \(\Sigma(t)\).
	I moti reali compatibili con il vincolo rendono stazionaria l'azione di Lagrangiana
	\[
		\mathcal{L}\big(\vec{x},\dot{\vec{x}}\big) = \frac{1}{2}\dot{\vec{x}} \sdot M\,\dot{\vec{x}} - U\big(\vec{x}\big)
	\]
	rispetto a variazioni in \(\mathcal{V}_{\vec{x}}\Big(\mathcal{M}_{t_1,t_2}^{\Sigma(t)}\big(\vec{\x}_1,\vec{\x}_2\big)\Big)\).
\end{teor}

\begin{oss}
	Se \(\vec{\j}(q_1,\ldots,q_s,t\big)\) è una parametrizzazione di \(\Sigma(t)\), le equazioni del moto sul vincolo nelle coordinate \(q_1,\ldots,q_s\), altro non sono che le equazioni di Eulero-Lagrange per la lagrangiana ridotta
	\[
		\tilde{\mathcal{L}}\big(\vec{q},\dot{\vec{q}},t) = \mathcal{L}\big(\vec{x},\dot{\vec{x}}\big)\bigg|_{\substack{\vec{x}=\vec{\j}(\vec{q},t)\\\dot{\vec{x}}=\vec{\y}(\vec{q},\dot{\vec{q}},t)}},
	\]
	dove
	\[
		\vec{\y}\big(\vec{q},\dot{\vec{q}},t\big) = \frac{\pd\vec{\j}}{\pd\utilde{q}}\big(\vec{q},t\big) \sdot \dot{\utilde{q}} + \frac{\pd\vec{\j}}{\pd t}\big(\vec{q},t\big).
	\]
\end{oss}

\begin{ese}
	Consideriamo un punto materiale vincolato ad una guida circolare, di raggio \(l\), che ruota uniformemente con pulsazione \(\w\). Il punto è soggetto alla forza di gravità e ad una reazione vincolare \(\vec{R}\) che supponiamo soddisfi il principio di D'Alembert. La forza conservativa attiva avrà potenziale \(U\big(\vec{x}\big)=m\,g\,z\). Passando in coordinate adattate al vincolo avremo
	\[
		\vec{x} = \vec{\j}(\q,t) = l\, 	\begin{pmatrix}
			\sin\q\,\cos(\w\,t) \\
			\sin\q\,\sin(\w\,t) \\
			-\cos\q
		\end{pmatrix}
	\]
	che dipende esplicitamente dal tempo. Da cui
	\[
		\dot{\vec{x}}(t) = l\,\dot{\q}\, \begin{pmatrix}\cos\q\,\cos(\w\,t)\\\cos\q\,\sin(\w\,t)\\\sin\q\end{pmatrix} + \w\,l\, \begin{pmatrix}-\sin\q\,\sin(\w\,t)\\\sin\q\,\cos(\w\,t)\\0\end{pmatrix} = \vec{\y}(\q,\dot{\q},t)
	\]
	Possiamo quindi scrivere la lagrangiana ridotta
	\[
		\tilde{\mathcal{L}}(\q,\dot{\q},t) = \frac{m}{2}\,\abs{\dot{\vec{x}}}^2-m\,g\,z\bigg|_{\substack{\vec{x}=\vec{\j}(\q,t)\\\dot{\vec{x}}=\vec{\y}(\q,\dot{\q},t)}}
	\]
	Ora, ricordando che \(\abs{\vec{a}+\vec{b}}^2 = \abs{\vec{a}}^2+\abs{\vec{b}}^2+2\vec{a}\sdot\vec{b}\), avremo
	\[
		\abs{\dot{\vec{x}}}^2 = l^2 \dot{\q}^2 + \w^2 l^2 \sin^2\q,
	\]
	da cui
	\[
		\tilde{\mathcal{L}}(\q,\dot{\q},t) = \frac{m}{2}\,( l^2 \dot{\q}^2 + \w^2 l^2 \sin^2\q) + m\,g\,l\,\cos\q,
	\]
	che osserviamo essere, in questo caso specifico, indipendente dal tempo.
	Applichiamo le equazioni di Eulero-Lagrange per trovare l'equazione del moto sul vincolo:
	\[
		\frac{\dd}{\dd t}\frac{\pd\tilde{\mathcal{L}}}{\pd\vec{\h}}\big(\q(t),\dot{\q}(t)\big) = \frac{\dd}{\dd t}(m\,l^2\,\dot{q}) = m\,l^2 \ddot{\q}(t),
	\]
	e
	\[
		\frac{\pd\tilde{\mathcal{L}}}{\pd\vec{\q}}\big(\q(t),\dot{\q}(t)\big) = m\,l^2 \left(\w^2 \sin\q(t)\,\cos\q(t)-\frac{g}{l}\sin\q(t)\right).
	\]
	Da cui
	\[
		m\,l^2 \ddot{\q} = m\,l^2 \sin\q\,\left(\w^2\cos\q- \frac{g}{l}\right) = -V'(\q).
	\]
	L'equazione del moto ammette una grandezza conservata
	\[
		E = \frac{m\,l^2}{2}\,\dot{\q}^2 + V(\q)
	\]
	la quale è conservata poiché, come al solito
	\[
		\frac{\dd}{\dd t}E = m\,l^2 \dot{\q}\,\ddot{\q} + V'(\q)\,\dot{\q} = 0 \iff \dot{\q}\,\big(m\,l^2\ddot{\q}+V'(\q)\big) = 0,
	\]
	che è vero per l'equazione del moto.
	Il moto si ottiene quindi per quadratura dalle solite relazioni
	\[
		\dot{\q} = \pm \sqrt{\frac{2}{m\,l^2}\big(E-V(\q)\big)} \qquad\text{e}\qquad t-t_0 = \int_{\q_0}^{\q(t)} \frac{\dd\q}{\sqrt{\frac{2}{m\,l^2}\big(E-V(\q)\big)}}
	\]
	Da questo punto in poi si può procedere con uno studio qualitativo del moto come visto nel capitolo precedente.
\end{ese}

\begin{oss}
	A differenza dell'esempio precedente, non sempre si arriva a dei risultati integrabili per quadrature.
	L'energia \(E\) conservata in questo esercizio è un caso particolare della cosiddetta \emph{energia generalizzata} che è una grandezza conservata per le equazioni di Eulero-Lagrange associata a qualsiasi lagrangiana \(\mathcal{L}\big(\vec{q},\dot{\vec{q}}\big)\) indipendente dal tempo.
	In questo caso, la conservazione dell'energia è conseguenza dell'indipendenza dal tempo della lagrangiana associata al sistema, oppure, come talvolta si dice, dell'invarianza di \(\mathcal{L}\) rispetto a traslazioni temporali.
\end{oss}

\begin{defn}{Energia generalizzata}{energiaGeneralizzata}\index{Energia generalizzata}
	Consideriamo un sistema di coordinate \(\vec{q}\) adattate al vincolo. L'espressione dell'\emph{energia generalizzata} è data da
	\[
		E = \dot{\vec{q}} \sdot \frac{\pd\mathcal{L}}{\pd\dot{\vec{q}}}\big(\vec{q},\dot{\vec{q}}\big) - \mathcal{L}\big(\vec{q},\dot{\vec{q}}\big).
	\]
\end{defn}

\begin{oss}
	Se \(\mathcal{L}\big(\vec{q},\dot{\vec{q}}\big) = \frac{1}{2}\dot{\vec{q}}\sdot M\big(\vec{q}\big)\,\dot{\vec{q}} - V\big(\vec{q}\big)\), l'espressione dell'energia generalizzata sarà
	\[
		E = \dot{\vec{q}}\sdot \frac{\pd\mathcal{L}}{\pd\dot{\vec{q}}} - \frac{1}{2}\dot{\vec{q}}\sdot M\big(\vec{q}\big)\,\dot{\vec{q}} - V\big(\vec{q}\big) = V\big(\vec{q}\big) + \frac{1}{2}\dot{\vec{q}}\sdot M\big(\vec{q}\big)\,\dot{\vec{q}} = U+K, 
	\]
	ovvero l'energia generalizzata è un'energia meccanica "standard"
\end{oss}

\begin{prop}{Energia generalizzata conservata dalle equazioni di E-L}{energiaGeneralizzataConservata}
	L'energia generalizzata è una grandezza conservata se \(\vec{q}(t)\) soddisfa le equazioni di Eulero-Lagrange.
\end{prop}

\begin{proof}
	Dobbiamo mostrare che l'espressione dell'energia generalizzata è un integrale primo se \(\vec{q}(t)\) risolve le equazioni di Eulero-Lagrange, ovvero se
	\[
		\frac{\dd}{\dd t}\frac{\pd\mathcal{L}}{\pd\dot{\vec{q}}}\big(\vec{q}(t),\dot{\vec{q}}(t)\big) = \frac{\pd\mathcal{L}}{\pd\vec{q}}\big(\vec{q}(t),\dot{\vec{q}}(t)\big).
	\]
	Calcoliamo quindi la derivata di \(E\) rispetto al tempo e mostriamo che è nulla:
	\[
		\frac{\dd}{\dd t}E = \ddot{\vec{q}}\sdot \frac{\pd\mathcal{L}}{\pd\dot{\vec{q}}} + \dot{\vec{q}}\sdot \frac{\dd}{\dd t}\frac{\pd\mathcal{L}}{\pd\dot{\vec{q}}} - \frac{\pd\mathcal{L}}{\pd\vec{q}}\sdot \dot{\vec{q}} - \frac{\pd\mathcal{L}}{\pd\dot{\vec{q}}}\sdot\ddot{\vec{q}} = \dot{\vec{q}}\sdot \left[\frac{\dd}{\dd t}\frac{\pd\mathcal{L}}{\pd\dot{\vec{q}}}-\frac{\pd\mathcal{L}}{\pd\vec{q}}\right] = 0,
	\]
	dove l'espressione in parentisi quadra è nulla poiché \(\vec{q}(t)\) risolve le equazioni di Eulero-Lagrange.
\end{proof}