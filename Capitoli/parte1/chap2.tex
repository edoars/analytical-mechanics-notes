%!TEX root = ../../dom.tex
%%%%%%%%%%%%%%%%%%%%%%%%%%%%%%%%%%%%%%%%%
%
%LEZIONE 03/03/2017 - PRIMA SETTIMANA (2)
%
%%%%%%%%%%%%%%%%%%%%%%%%%%%%%%%%%%%%%%%%%
\chapter{Analisi qualitativa del moto e stabilità}

Salvo precisazioni ulteriori, faremo riferimento a
\[\label{eq:sistema}
	M\,\ddot{\vec{R}} = -\frac{\pd}{\pd\vec{R}}U\big(\vec{R}\big)\tag{\(\star\)}
\]
come al \emph{sistema} o al sistema \eqref{eq:sistema}.
%%%%%%%%%%%%%%
%INTRODUZIONE%
%%%%%%%%%%%%%%
\section{Introduzione}

In generale, quando diciamo che \(M\,\ddot{\vec{R}} = -\frac{\pd}{\pd\vec{R}}U\big(\vec{R}\big)\) è non risolubile, intendiamo che non lo è per dati iniziali generici.
D'altronde è quasi sempre possibile identificare dati iniziali "speciali" che ci permettono di esplicitare una soluzione. In questi casi è inoltre possibile eseguire un'analisi qualitativa del problema negli intorni di tali punti.

\begin{defn}{Punto di equilibrio}{puntoEquilibrio}\index{Punto di equilibrio}
	Definiamo \emph{punto di equilibrio} del sistema \eqref{eq:sistema} un punto \(\vec{R}_{eq}\) tale che
	\[
		\frac{\pd}{\pd\vec{R}}U\big(\vec{R}_{eq}\big) = 0,
	\]
	ovvero un punto critico del potenziale.
\end{defn}

\begin{oss}
	Spesso, la nozione di punto di equilibrio viene discusso per un sistema del tipo
	\[
		\dot{\vec{x}} = \vec{G}\big(\vec{x}\big).
	\]
	In tal caso si dice che \(\vec{x}_{eq}\) è un punto di equilibrio se \(\vec{G}\big(\vec{x}_{eq}\big)=0\).
	
	Le equazioni di Newton sono un caso particolare di tale sistema, infatti le equazioni
	\[
		M\,\ddot{\vec{R}} = F\big(\vec{R},\dot{\vec{R}}\big) \iff
		\begin{cases}
			\dot{\vec{R}} = \vec{V} \\
			\dot{\vec{V}} = M^{-1}F\big(\vec{R},\vec{V}\big)
		\end{cases}
	\]
	possono essere viste attraverso \(\dot{\vec{x}}=\vec{G}\big(\vec{x}\big)\) ponendo
	\[
		\vec{x} = \begin{pmatrix}\vec{R}\\\vec{V}\end{pmatrix}
	\]
\end{oss}

\begin{ese}
	Un caso particolare è quello in cui, posto un punto di equilibrio \(\vec{R}_{eq}\), si hanno i seguenti dati iniziali:
	\[
		\vec{R}(0) = \vec{R}_{eq} \qquad\text{e}\qquad \dot{\vec{R}}(0) = \vec{0}.
	\]
	In tal caso si ha che \(\vec{R}(t) \equiv \vec{R}_{eq}\) è soluzione di \eqref{eq:sistema}.
\end{ese}

Un approccio naturale, che andremo ad analizzare in questo capitolo, è quello di studiare il comportamento nel sistema, se scegliamo un punto vicino a quello di equilibrio come punto iniziale.
In particolare osserveremo che tale comportamento definisce punti di equilibrio "stabili" o "instabili"; a seconda di quanto dati iniziali vicini a \(\vec{R}_{eq}\) rimangono vicini ad esso.

\begin{defn}{Stabilità secondo Ljapunov}{stabilitàLjapunov}\index{Stabilità!secondo Ljapunov}
	Un punto di equilibrio \(\vec{R}_{eq}\) si dice \emph{stabile}, nel senso di Ljapunov, se per ogni \(\e>0\) esiste un \(\d=\d_\e>0\) tale che, se
	\[
		\sqrt{\abs{\vec{R}_0-\vec{R}_{eq}}^2 +\t_0^2\abs{\vec{V}_0}^2} \le \d,
	\]
	allora
	\[
		\sqrt{\abs{\vec{R}(t)-\vec{R}_{eq}}^2+\t_0^2\abs{\vec{V}(t)}^2} \le \e,\,\fa t \ge 0.
	\]
\end{defn}

\begin{notz}
	La quantità \(\t_0\) si definisce \emph{unità temporale} e serve ad aggiustare le unità di misura.
\end{notz}

\begin{oss}
	Viceversa \(\vec{R}_{eq}\) si dice \emph{instabile} se non è stabile, cioè se esiste un \(\e>0\) tale che per ogni \(\d>0\) esistono \(\vec{R}_0,\vec{V}_0\) e \(t_1>0\) con
	\[
		\sqrt{\abs{\vec{R}_0-\vec{R}_{eq}}^2 +\t_0^2\abs{\vec{V}_0}^2} \le \d,
	\]
	tali che
	\[
		\sqrt{\abs{\vec{R}(t_1)-\vec{R}_{eq}}^2+\t_0^2\abs{\vec{V}(t_1)}^2} > \e.
	\]
\end{oss}

\begin{teor}{di Dirichlet}{teoremaDirichlet}\index{Teorema!di Dirichlet}
	Sia \(\vec{R}_{eq}\) un punto di minimo locale stretto per \(U\big(\vec{R}\big)\).
	Allora \(\vec{R}_{eq}\) è stabile.
\end{teor}
%%%%%%%%%%%%%%%%%%%%%%%%%%%%%%%%%%%%%%%%%%%
%
%LEZIONE 06/03/2017 - SECONDA SETTIMANA (1)
%
%%%%%%%%%%%%%%%%%%%%%%%%%%%%%%%%%%%%%%%%%%%
\begin{proof}
	Per dimostrare che \(\vec{R}_{eq}\) è stabile, dobbiamo mostrare che per dati iniziali sufficientemente vicini a \(\big(\vec{R}_{eq},\vec{0}\big)\), i valori \(\big(\vec{R},\dot{\vec{R}}\big)\) continuano a restarvi vicini nel tempo.
	
	Consideriamo la seguente funzione
	\[
		W\big(\vec{R},\dot{\vec{R}}\big) = \frac{1}{2}\,\dot{\vec{R}}\sdot M\,\dot{\vec{R}} + U\big(\vec{R}\big) - U\big(\vec{R}_{eq}\big)
	\]
	osservando che \(W\big(\vec{R}_{eq},\vec{0}\big)=0\).
	Dal momento che, per ipotesi, \(\vec{R}_{eq}\) un minimo stretto di \(U\), sul bordo di un intorno di \(\big(\vec{R}_{eq},\vec{0}\big)\), la quantità \(W\) sarà sempre positiva. Formalmente, definito
	\[
		I_\e = \Set{\big(\vec{R},\dot{\vec{R}}\big) : \sqrt{\abs{\vec{R}-\vec{R}_{eq}}^2+\t_0^2\abs{\dot{\vec{R}}}^2} \le \e}
	\]
	avremo che, per ogni \(\e>0\) sufficientemente piccolo,
	\[
		E_\e:=\min_{(\vec{R},\dot{\vec{R}}) \in \pd I_\e} W\big(\vec{R},\dot{\vec{R}}\big) > 0
	\]
	Inoltre \(W\big(\vec{R},\dot{\vec{R}}\big) \to 0\) per \(\big(\vec{R},\dot{\vec{R}}\big) \to \big(\vec{R}_{eq},\vec{0}\big)\); in particolare esisterà un \(\d_\e\) tale che
	\[
		\max_{(\vec{R},\dot{\vec{R}})\in I_{\d_\e}} W\big(\vec{R},\dot{\vec{R}}\big) \le \frac{E_\e}{2}.
	\]
	Quindi, dati iniziali in \(I_{\d_\e}\), generano moti tali che
	\[
		W\big(\vec{R}(t),\dot{\vec{R}}(t)\big) \le \frac{E_\e}{2} \qquad\text{se }\big(\vec{R}(t),\dot{\vec{R}}(t)\big) \in I_{\d_\e},\,\fa t \ge 0.
	\]
	Ciò significa che \(\big(\vec{R},\dot{\vec{R}}\big)\) non può raggiungere \(\pd I_\e\) su cui \(W \ge E_\e\) e resteranno pertanto vicini al punto di equilibrio.
\end{proof}

\begin{oss}
	Ricordiamo che se \(\vec{R}_{eq}\) un punto critico per \(U\big(\vec{R}\big)\) e consideriamo la matrice Hessiana in \(\vec{R}_{eq}\)
	\[
		\big(H_0\big)_{i j} = \frac{\pd^2}{\pd R_i \pd R_j}U\big(\vec{R}_{eq}\big),
	\]
	abbiamo le seguenti possibilità:
	\begin{itemize}
		\item Se \(H_0\) è definita strettamente positiva, allora \(\vec{R}_{eq}\) è un minimo locale stretto di \(U\).
		\item Se \(H_0\) è definita strettamente negativa, allora \(\vec{R}_{eq}\) è un massimo locale stretto di \(U\).
		\item Se \(H_0\) ha autovalori positivi e negativi, allora \(\vec{R}_{eq}\) è un punto di sella di \(U\).
	\end{itemize}
\end{oss}

\begin{notz}
	Con punto di sella non degenere, intendiamo che vi è un autovalore negativo dell'Hessiano di \(U\) in \(\vec{R}_{eq}\).
\end{notz}

\begin{prop}{Condizione di instabilità}{condizionaInstabilità}
	Sia \(\vec{R}_{eq}\) un punto di massimo locale o di sella non degenere per \(U\big(\vec{R}\big)\).
	Allora \(\vec{R}_{eq}\) è instabile.
\end{prop}

\begin{proof}
	Per semplicità discuteremo la dimostrazione per moti in una dimensione.
	L'equazione del moto a cui faremo riferimento sarà pertanto
	\[
		m\,\ddot{x} = -U'(x).
	\]
	Supponiamo che \(x_{eq}\) sia un massimo stretto non degenere\graffito{Il caso dei punti di sella non è applicabile}, ovvero tale che
	\[
		U'(x_{eq}) = 0 \qquad\text{e}\qquad U''(x_{eq}) < 0.
	\]
	Sfruttando queste proprietà, sviluppiamo \(-U'(x)\) in serie di Taylor attorno a \(x_{eq}\):
	\[
		-U'(x) = -U'(x_{eq}) - U''(x_{eq})(x-x_{eq}) + R(x,x_{eq}) = -U''(x_{eq})(x-x_{eq})+R(x,x_{eq}),
	\]
	dove \(R(x,x_{eq}) \in \bO\big((x-x_{eq})^2\big)\).
	In particolare \(U''(x_{eq})\) è una costante negativa, scriviamola come
	\[
		U''(x_{eq}) = -m\,\t_0^{-2} \qquad\text{con }\t_0 = \sqrt{\frac{m}{-U''(x_{eq})}}.
	\]
	Sostituendo nell'equazione del moto otteniamo
	\begin{equation}
		m\,\ddot{x} = m\,\t_0^{-2}(x-x_{eq}) + R(x,x_{eq}) \implies \ddot{x} = \t_0^{-2}(x-x_{eq})+\frac{R(x,x_{eq})}{m}.\tag{1}
	\end{equation}
	Trascurando il resto, otteniamo l'equazione linearizzata
	\[
		\ddot{x} = \t_0^{-2}(x-x_{eq}),
	\]
	la cui soluzione generale è
	\[
		x(t)-x_{eq} = a\,e^{t/\t_0} + b\,e^{-t/\t_0}.
	\]
	Imponendo i dati iniziali \(x(0)=x_0\) e \(\dot{x}(0)=v_0\), otteniamo
	\[
		\begin{cases}
			x_0-x_{eq} = a+b \\
			\t_0v_0 = a-b
		\end{cases} \implies
		\begin{cases}
			a = \frac{1}{2}(x_0-x_{eq}+\t_0 v_0) \\
			b = \frac{1}{2}(x_0-x_{eq}-\t_0 v_0)
		\end{cases}
	\]
	Quindi
	\[
		x(t)-x_{eq} = \frac{1}{2}(x_0-x_{eq}+\t_0 v_0)\,e^{t/\t_0} + \frac{1}{2}(x_0-x_{eq}-\t_0 v_0)\,e^{-t/\t_0}.
	\]
	Per dimostrare che \(x_{eq}\) è un punto instabile, vogliamo mostrare che esistono dati iniziali arbitrariamente vicini a \((x_{eq},0)\) che fuoriescono, in un tempo finito, da un intorno
	\[
		I_{\d_\e} = \Set{(x,\dot{x}): \sqrt{\abs{x-x_{eq}}^2+\t_0^2 \abs{\dot{x}}^2} \le \d_\e}
	\]
	Scegliamo i dati iniziali \((x_{eq}+\d',0)\) con \(\d'>0\). Sulla base della soluzione generale ricavata prima, definiamo la seguente funzione
	\[
		g(t) := x(t)-x_{eq} + \frac{\t_0}{\sqrt{1-\e}}\dot{x}(t).
	\]
	Scriviamo la sua derivata e stimiamola in termini di \(g\) per riportarci ad una disequazione differenziale:
	\[
		\frac{\dd}{\dd t}g(t) = \dot{x}(t) + \frac{\t_0}{\sqrt{1-\e}}\ddot{x}(t) \overset{1}{=} \dot{x}(t) + \frac{\t_0}{\sqrt{1-\e}} \left( \t_0^{-2}(x-x_{eq}) + \frac{R(x,x_{eq})}{m} \right).
	\]
	Possiamo stimare il resto con il termine di ordine inferiore:
	\[
		\fa \e >0\,\ex \d_\e: \abs{x-x_{eq}} \le \d_\e \implies \frac{\abs{R(x,x_{eq})}}{m} \le \e\,\t_0^{-2}\abs{x-x_{eq}}.
	\]
	Quindi, fintanto che \(\abs{x-x_{eq}} \le \d_\e\) e ricordando che \(x-x_{eq} > 0\), otteniamo che
	\[
		\frac{\dd}{\dd t}g(t) \ge \dot{x} + \frac{\t_0}{\sqrt{1-\e}}\frac{1-\e}{\t_0^2}(x-x_{eq}) = \frac{\sqrt{1-\e}}{\t_0} \left( x-x_{eq}+ \frac{\t_0}{\sqrt{1-\e}}\dot{x} \right),
	\]
	ovvero
	\[
		\frac{\dd}{\dd t}g(t) \ge \frac{\sqrt{1-\e}}{\t_0}g(t).
	\]
	Il lemma di Gronwall afferma che se esiste \(g\) che soddisfa tale disequazione differenziale, allora tale soluzione viene limitata dal basso dalla soluzione dell'equazione differenziale associata.
	Per trovare quest'ultima ricordiamo che \(g\) è positiva dall'istante iniziale, pertanto possiamo dividere per \(g\) e risolvere per separazione di variabili:
	\[
		\begin{split}
			g'(t) = \frac{\sqrt{1-\e}}{\t_0}g(t) & \implies \frac{g'(t)}{g(t)} = \frac{\sqrt{1-\e}}{\t_0} \implies \frac{\dd}{\dd t}\ln\big(g(t)\big) = \frac{\sqrt{1-\e}}{\t_0}\\
			& \implies \ln\big(g(t)\big) = \frac{\sqrt{1-\e}}{\t_0}t + \ln(g_0)\\
			& \implies g(t) = g_0 e^{\frac{\sqrt{1-\e}}{\t_0}t}.
		\end{split}
	\]
	Da cui, applicando il lemma,
	\[
		g(t) \ge g_0 e^{\frac{t\,\sqrt{1-\e}}{\t_0}}.
	\]
	Da questa stima, ricordando la definizione di \(g\) e la banale disuguaglianza \(\sqrt{2}\sqrt{a^2+b^2} \ge a+b\), otteniamo
	\[
		\sqrt{\frac{2}{1-\e}}\sqrt{\abs{x(t)-x_{eq}}^2+\t_0^2\abs{x(t)}^2} \ge x(t)-x_{eq} + \frac{\t_0}{\sqrt{1-\e}}\dot{x}(t) = g(t) \ge g_0 e^{\frac{t\,\sqrt{1-\e}}{\t_0}},
	\]
	che è certamente maggiore di \(\e\) per \(t\) sufficientemente grande.
\end{proof}
%%%%%%%%%%%%%%%%%%%%%
%SISTEMI CON ATTRITO%
%%%%%%%%%%%%%%%%%%%%%
\section{Sistemi con attrito}

Vorremmo ora considerare l'effetto dell'attrito sulla stabilità dei punti di equilibrio.
L'equazione a cui facciamo riferimento in questo caso è
\[
	M\,\ddot{\vec{R}} = -\frac{\pd U}{\pd \vec{R}}\big(\vec{R}\big) - \g\,\dot{\vec{R}}.\tag{\(\star\star\)}
\]
In particolare si dimostra che i punti di massimo o di sella non degeneri rimangono instabili come nel caso senza attrito; viceversa, i punti di minimo stretto non degeneri diventano asintoticamente stabili.

\begin{defn}{Punti asintoticamente stabili}{puntiAsintoticamenteStabili}\index{Punto di equilibrio!asintoticamente stabile}
	Un punto di equilibrio \(\vec{R}_{eq}\) per un sistema dotato di attrito \((\star\star)\), si dice \emph{asintoticamente stabile} se è stabile ed esiste \(\d\) per cui, dati iniziali in 
	\[
		I_\d = \Set{\big(\vec{R},\dot{\vec{R}}\big) : \sqrt{\abs{\vec{R}-\vec{R}_{eq}}^2+\t_0^2\abs{\dot{\vec{R}}}^2} \le \d},
	\]
	generano moti tali che
	\[
		\big(\vec{R}(t),\dot{\vec{R}}(t)\big) \longrightarrow \big(\vec{R}_{eq},\vec{0}\big) \qquad\text{per }t \to +\infty.
	\]
\end{defn}

\begin{teor}{di Dirichlet per sistemi con attrito}{teoremaDirichletAttrito}
	Sia \(\vec{R}_{eq}\) un punto di minimo locale stretto non degenere per \(U\).
	Allora \(\vec{R}_{eq}\) è un punto di equilibrio asintoticamente stabile per
	\[
		M\,\ddot{\vec{R}} = -\frac{\pd U}{\pd\vec{R}}\big(\vec{R}\big) - \g\dot{\vec{R}} \qquad\text{con }\g > 0.
	\]
\end{teor}

\begin{proof}
	Nuovamente per semplicità, discuteremo la dimostrazione per moti in una dimensione.
	L'equazione a cui faremo riferimento sarà pertanto
	\[
		m\,\ddot{x}=-U'(x)-\g\,\dot{x}.
	\]
	Supponiamo che \(x_{eq}\) sia un minimo stretto non degenere, per cui
	\[
		U'(x_{eq}) = 0 \qquad\text{e}\qquad U''(x_{eq}) > 0.
	\]
	\(U''(x_{eq})\) è una costante positiva, quindi, analogamente alla dimostrazione precedente, definiamo
	\[
		U''(x_{eq}) = \w_0^2 m \qquad\text{con } \w_0 = \sqrt{\frac{U''(x_{eq})}{m}}.
	\]
	Sviluppando \(U'(x)\) con Taylor e sostituendo nell'equazione generale otteniamo
	\[
		m\,\ddot{x} = -U''(x_{eq})(x-x_{eq}) + R(x,x_{eq}) - \g\,\dot{x} \implies \ddot{x} = -w_0^2(x-x_{eq}) - \frac{\g}{m}\,\dot{x} + \frac{R(x,x_{eq})}{m}.
	\]
	Da cui otteniamo l'equazione linearizzata
	\[
		\ddot{x} + \frac{\g}{m}\,\dot{x} + \w_0^2 (x-x_{eq}) = 0,
	\]
	che corrisponde all'equazione dell'oscillatore armonico smorzato.
	
	Risolvendo esplicitamente l'equazione e stimandola in maniera opportuna, possiamo costruire una funzione che ci dia la tesi nel caso generale come per la \hyperref[pr:condizionaInstabilità]{proposizione sulla condizione di instabilità}
\end{proof}
%%%%%%%%%%%%%%%%%%%%%%%%%%%%%%%%%%%%%%%%%%%
%
%LEZIONE 08/03/2017 - SECONDA SETTIMANA (2)
%
%%%%%%%%%%%%%%%%%%%%%%%%%%%%%%%%%%%%%%%%%%%
%%%%%%%%%%%%%%%%%%%%%%
%PICCOLE OSCILLAZIONI%
%%%%%%%%%%%%%%%%%%%%%%
\section{Piccole oscillazioni}

Torniamo nel caso conservativo privo di attrito e andiamo a studiare con attenzione il caso del moto vicino ad un punto di equilibrio stabile.

Abbiamo dimostrato che \(\vec{R}_{eq}\) è un punto di equilibrio stabile se è un minimo stretto non degenere per \(U\).
Sviluppando \(U\) nella sua serie di Taylor attorno a \(\vec{R}_{eq}\) e sostituendo nell'equazione del moto, otteniamo
\[
	M\,\ddot{\vec{R}} = -\underbrace{\frac{\pd U}{\pd \vec{R}}\big(\vec{R}_{eq})}_{= 0} - H_0 \big(\vec{R}-\vec{R}_{eq}\big) + \vec{C}\big(\vec{R},\vec{R}_{eq}\big),
\]
dove \(\vec{C}\big(\vec{R},\vec{R}_{eq}\big)\) è il resto di Taylor del secondo ordine.

\begin{oss}
	Per definizione, \(\vec{R}_{eq}\) è non degenere se la matrice Hessiana \(H_0\) è definita positiva, ovvero se il minimo autovalore \(\l_{\min}\) di \(H_0\) è strettamente positivo.
	Una definizione analoga è la seguente:
	\[
		H_0 \ge \l_{\min}Id \qquad\text{ovvero } \vec{u}\sdot H_0\vec{u} \ge \l_{\min}\abs{\vec{u}}^2.
	\]
\end{oss}

Per \hyperref[th:teoremaDirichlet]{Dirichlet}, dati iniziali vicini a \(\vec{R}_{eq}\) stabile generano un moto che vi rimane vicino; ciò vuol dire che il resto di Taylor potrà sempre essere stimato tramite l'elemento del primo ordine a meno di un \(\e\). Infatti
\[
	\abs{\vec{C}\big(\vec{R},\vec{R}_{eq}\big)} \le K\,\abs{\vec{R}-\vec{R}_{eq}}^2 \le \e\,\l_{\min}\abs{\vec{R}-\vec{R}_{eq}}
\]
fintanto che \(\abs{\vec{R}-\vec{R}_{eq}}\) è abbastanza piccolo.
D'altronde nel caso di \(\vec{R}_{eq}\) stabile, \(\abs{\vec{R}-\vec{R}_{eq}}\) è piccolo ad ogni istante \(t\ge 0\).
Quindi l'equazione linearizzata
\[
	M\,\ddot{\vec{R}} = -H_0\big(\vec{R}-\vec{R}_{eq}\big),\tag{\(\star\)}
\]
è vicina all'equazione iniziale in ogni istante \(t\ge 0\).

Cerchiamo ora di risolvere l'equazione linearizzata.
Definiamo\graffito{la definizione è lecita in quanto \(M\) è una matrice diagonale con elementi positivi}
\[
	\vec{X} = M^{\frac{1}{2}}\big(\vec{R}-\vec{R}_{eq}\big),
\]
dove \(M^{\frac{1}{2}}\) è la matrice diagonale con elementi \(\sqrt{M_{ii}}\).
Ora. sfruttando la \((\star)\),
\[
	\begin{split}
		\ddot{\vec{X}} & = M^{\frac{1}{2}}\ddot{\vec{R}} = -M^{\frac{1}{2}}M^{-1}H_0\big(\vec{R}-\vec{R}_{eq}\big) = -M^{-\frac{1}{2}}H_0\big(\vec{R}-\vec{R}_{eq}\big)\\
		& = -(M^{-\frac{1}{2}}H_0 M^{-\frac{1}{2}}) M^{\frac{1}{2}}\big(\vec{R}-\vec{R}_{eq}\big)\\
		& = -H\,\vec{X},
	\end{split}
\]
dove \(H=M^{-\frac{1}{2}}H_0 M^{-\frac{1}{2}}\) è simmetrica e definita strettamente positiva. Da cui
\[
	M\,\ddot{\vec{R}} = -H_0 \big(\vec{R}-\vec{R}_{eq}\big) \iff M^{\frac{1}{2}}\ddot{\vec{X}} = -H_0 M^{-\frac{1}{2}} \vec{X}.
\]

\begin{oss}
	Ingenuamente si potrebbe tentare di lavorare direttamente su
	\[
		\ddot{\vec{R}} = -M^{-1}H_0\big(\vec{R}-\vec{R}_{eq}\big).
	\]
	Ma, in generale, la matrice \(M^{-1} H_0\) non è simmetrica.
\end{oss}
\noindent
\(H\), in quanto simmetrica e definita strettamente positiva, è diagonalizzabile.
Pertanto esiste una base ortonormale \(\vec{v}_1,\ldots,\vec{v}_n\) di \(H\), tale che
\[
	H\,\vec{v}_i = \w_i^2 \vec{v}_i \qquad\text{con }\w_i \neq 0,
\]
dove gli \(\w_i^2\) sono gli autovalori associati; questa notazione serve a sottolinearne la positività.

\begin{defn}{Frequenze normali}{freqenzeNormali}\index{Frequenze normali}
	Sia \(\vec{R}_{eq}\) un punto di equilibrio stabile per il sistema.
	Si definiscono \emph{frequenza normale} delle piccole oscillazioni, gli autovalori \(\w_i^2\) associati alla matrice
	\[
		H=M^{-\frac{1}{2}}H_0 M^{-\frac{1}{2}}. 
	\]
\end{defn}

\begin{notz}
	Gli autovalori \(\w_i^2\) prendono anche il nome di frequenze proprie o caratteristiche.
\end{notz}

\noindent
Esprimendo i valori di \(\vec{X}\) tramite la base diagonalizzante per \(H\), avremo
\[
	\vec{X}(t) = y_1(t)\vec{v}_1 + \ldots + y_n(t)\vec{v}_n,
\]
dove
\[
	\ddot{y}_i(t) = -\w_i^2 y_i(t).
\]
Risolvendo questa equazione differenziale si ottiene
\[
	y_i(t) = y_i(0)\,\cos(\w_i t) + \frac{\dot{y}_i(0)}{\w_i}\,\sin(\w_i t) = a_i \cos(\w_i t + \j_i).
\]
Riportandoci in forma vettoriale:
\[
	\vec{X}(t) = a_1 \cos(\w_1 t + \j_1)\,\vec{v}_1 + \ldots + a_n \cos(\w_n t + \j_n)\,\vec{v}_n,
\]
da cui, ricordando la definizione di \(\vec{X}\), otteniamo infine
\[
	\vec{R}(t) = \vec{R}_{eq} + M^{-\frac{1}{2}}\vec{X}(t) = \vec{R}_{eq}+ a_1 \cos(\w_1 t + \j_1)\,M^{-\frac{1}{2}}\vec{v}_1 + \ldots + a_n \cos(\w_n t + \j_n)\,M^{-\frac{1}{2}}\vec{v}_n
\]

\begin{oss}
	Per calcolare gli autovalori di \(H\), dobbiamo trovare le radici del suo polinomio caratteristico. Alternativamente possiamo ridurci ad \(H_0\):
	\[
		\begin{split}
			\det(H-\l\,Id) & = \det(M^{-\frac{1}{2}} H_0 M^{-\frac{1}{2}} - \l\,Id) = \det\big[M^{-\frac{1}{2}}(H_0-\l\,M)M^{-\frac{1}{2}}\big]\\
			& = \det(M^{-\frac{1}{2}})\,\det(H_0-\l\,M)\,\det(M^{-\frac{1}{2}}) = \frac{1}{\det M}\,\det(H_0-\l\,M).
		\end{split}
	\]
	In particolare, dal momento che \(\det M \neq 0\), avremo
	\[
		\det(H-\l\,Id) = 0 \iff \det(H_0-\l\,M) = 0.
	\]
\end{oss}

\begin{oss}
	Anche per l'espressione dell'azione degli autovettori di \(H\) possiamo ricondurci ad \(H_0\), infatti:
	\[
		H\,\vec{v}_i = \w_i^2 \vec{v}_i \iff M^{-\frac{1}{2}} H_0 M^{-\frac{1}{2}} \vec{v}_i = \w_i^2 \vec{v}_i \iff H_0 \vec{u}_i = \w_i^2 M\,\vec{u}_i,
	\]
	dove \(\vec{u}_i = M^{-\frac{1}{2}}\vec{v}_i\).
\end{oss}
%%%%%%%%%%
%ESERCIZI%
%%%%%%%%%%
\section{Esercizi}
\begin{exeN}[Esercitazione 08/03]
	Considerare il moto di un punto materiale, di massa \(m=1\), soggetto ad un potenziale
	\[
		U(x) = \frac{x^4}{4} + \a\,\frac{x^2}{2}.
	\]
	\begin{enumerate}
		\item Trovare, al variare di \(\a\in \R\setminus\{0\}\), i punti di equilibrio del sistema e studiarne la stabilità.
		\item Nei casi in cui \(x_{eq}\) è stabile, trovare l'equazione linearizzata associata e la frequenza normale delle piccole oscillazioni.
	\end{enumerate}
\end{exeN}

\begin{sol}
	\graffito{\(1)\)}Per definizione un punto \(x_{eq}\) è di equilibrio se \(U'(x_{eq}) = 0\).
	Ora
	\[
		U'(x) = x^3 + \a\,x = x\,(x^2+\a).
	\]
	Quindi
	\[
		U'(x) = 0 \iff 	\begin{cases}
			x = 0                        & \text{se }\a > 0   \\
			x = 0 \vee x = \pm\sqrt{-\a} & \text{se }\a < 0.
		\end{cases}
	\]
	Per cui i punti di equilibrio saranno \(x_1 = 0\) se \(\a>0\), oppure \(x_2=0,x_{3,4}=\pm\sqrt{-\a}\) se \(\a<0\). Valutiamone la stabilità:
	\[
		U''(x) = 3x^2+\a,
	\]
	quindi
	\begin{gather*}
		U''(x_1) = \a > 0 \qquad\text{se }\a > 0;\\
		U''(x_2) = \a < 0 \qquad\text{se }\a < 0;\\
		U''(x_{3,4}) = -3\a+\a = -2\a > 0 \qquad\text{se }\a < 0.
	\end{gather*}
	Applicando il \hyperref[th:teoremaDirichlet]{teorema di Dirichlet}, avremo che \(x_1,x_{3,4}\) sono punti di minimo e pertanto stabili per il sistema; mentre \(x_2\) è un punto di massimo e quindi instabile per il sistema.
	
	\graffito{\(2)\)}In generale sappiamo che l'equazione linearizzata associata ad un punto di equilibrio è data da
	\[
		M\,\ddot{\vec{x}} = -H_0\big(\vec{x}-\vec{x}_{eq}\big).
	\]
	Nel caso monodimensionale, \(H_0\) è semplicemente \(U''(x_{eq})\).
	Quindi per il caso \(\a>0\) avremo
	\[
		\ddot{x} = -\a\,x \qquad\text{e}\qquad \w^2 = \a.
	\]
	Mentre, per il caso \(\a<0\), avremo
	\begin{gather*}
		\ddot{x} = -(-2\a)(x-\sqrt{-\a}) \qquad\text{e}\qquad \w^2 = -2\a;\\
		\ddot{x} = -(-2\a)(x+\sqrt{-\a}) \qquad\text{e}\qquad \w^2 = -2\a.
	\end{gather*}
\end{sol}