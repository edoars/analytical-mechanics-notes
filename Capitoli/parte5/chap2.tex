%!TEX root = ../../main.tex
\chapter{Calcolo delle variazioni e trasformazioni canoniche}
%%%%%%%%%%%%%%%%%%%%%%%%%%%%%%%%%%%%%%%%%%%%%%
%
%LEZIONE 17/05/2017 - UNDICESIMA SETTIMANA (2)
%
%%%%%%%%%%%%%%%%%%%%%%%%%%%%%%%%%%%%%%%%%%%%%%
\section{Principio di variazione di Hamilton}

Ricordiamo che le equazioni di Eulero-Lagrange rendono stazionaria l'azione
\[
	A_{t_1,t_2}^{\mathcal{L}}\big[\vec{q}(t)\big] = \int_{t_1}^{t_2}\mathcal{L}\big(\vec{q}(t),\dot{\vec{q}}(t)\big)\,\dd t.
\]
D'altronde, sappiamo che tali equazioni sono equivalenti alle equazioni di Hamilton. Pertanto è lecito chiedersi quale azione renderanno stazionaria tali equazioni.

\begin{defn}{Azione di Hamilton}{azioneHamilton}\index{Azione di Hamilton}
	Consideriamo l'hamiltoniana \(\mathcal{H}=\mathcal{H}\big(\vec{q},\vec{p}\big)\). \emph{L'azione di Hamilton} associata ad \(\mathcal{H}\) è definita come
	\[
		S_{t_1,t_2}^{\mathcal{H}}\big[\big(\vec{q}(t),\vec{p}(t)\big)\big] = \int_{t_1}^{t_2}\big[\vec{p}(t)\sdot \dot{\vec{q}}(t)-\mathcal{H}\big(\vec{q}(t),\vec{p}(t)\big)\big]\,\dd t
	\]
\end{defn}

\begin{defn}{Variazioni di Hamilton}{variazioniHamilton}
	Definiamo una \emph{variazione} di \(\big(\vec{q}(t),\vec{p}(t)\big)\) come
	\[
		\big(\vec{q}(t),\vec{p}(t)\big) + \big(\vec{u}_\e(t),\vec{v}_\e(t)\big) = \big(\vec{q}(t)+\vec{v}_\e(t),\vec{p}(t)+\vec{v}_\e(t)\big),
	\]
	dove \(\vec{u}_\e,\vec{v}_\e\) sono tali che
	\[
		\begin{matrix}
			\vec{u}_0(t) \equiv \vec{0} \\
			\vec{v}_0(t) \equiv \vec{0}
		\end{matrix} \qquad\text{e}\qquad
		\begin{matrix}
			\vec{u}_\e(t_1) = \vec{u}_\e(t_2) = \vec{0} \\
			\vec{v}_\e(t_1) = \vec{v}_\e(t_2) = \vec{0}
		\end{matrix}
	\]
\end{defn}

\begin{notz}
	Abbiamo considerato \(\big(\vec{q}(t),\vec{p}(t)\big)\) sullo spazio dei moti in \(\R^{2n}\) sull'intervallo temporale \([t_1,t_2]\) ed estremi fissati
	\[
		\big(\vec{q}_1,\vec{p}_1\big) = \big(\vec{q}(t_1),\vec{p}(t_1)\big) \qquad\text{e}\qquad \big(\vec{q}_2,\vec{p}_2\big) = \big(\vec{q}(t_2),\vec{p}(t_2)\big).
	\]
\end{notz}

\begin{teor}{Stazionarietà del funzionale hamiltoniano}{stazionarietàFunzionaleHamiltoniano}
	Il moto \(\big(\vec{q}(t),\vec{p}(t)\big)\) inteso come nella notazione precedente, è soluzione delle equazioni di Hamilton se e soltanto se \(\big(\vec{q}(t),\vec{p}(t)\big)\) rende stazionaria l'azione
	\[
		S_{t_1,t_2}^{\mathcal{H}}\big[\big(\vec{q}(t),\vec{p}(t)\big)\big]
	\]
	rispetto a variazioni \(\big(\vec{q}(t)+\vec{v}_\e(t),\vec{p}(t)+\vec{v}_\e(t)\big)\).
\end{teor}

\begin{proof}
	La dimostrazione è analoga al caso lagrangiano. Forniamo una traccia di dimostrazione che mostra come manipolare il funzionale d'azione.
	Ricordiamo che la condizione di stazionarietà dell'azione è data da
	\[
		\frac{\dd}{\dd\e} S_{t_1,t_2}^{\mathcal{H}}\big[\big(\vec{q}(t)+\vec{v}_\e(t),\vec{p}(t)+\vec{v}_\e(t)\big)\big]\bigg|_{\e = 0} = 0.
	\]
	Sviluppiamo \(S_{t_1,t_2}^{\mathcal{H}}\) per ottenere
	\[
		\frac{\dd}{\dd\e}\int_{t_1}^{t_2}\Big[\big(\vec{p}(t)+\vec{v}_\e(t)\big) \sdot \big(\dot{\vec{q}}(t)+\dot{\vec{u}}_\e(t)\big) - \mathcal{H}\big(\vec{q}(t)+\vec{u}_\e(t),\vec{p}(t)+\vec{v}_\e(t)\big)\Big]\,\dd t \bigg|_{\e=0}.
	\]
	\'E conveniente definire la seguente notazione:
	\[
		\frac{\dd}{\dd\e}\vec{u}_\e(t)\bigg|_{\e=0} = \vec{z}(t) \qquad\text{e}\qquad \frac{\dd}{\dd\e}\vec{v}_\e(t)\bigg|_{\e=0}=\vec{w}(t).
	\]
	Da cui, portando la derivata dentro l'integrale e sostituendo, si ottiene
	\[
		\int_{t_1}^{t_2}\Big[\vec{w}(t) \sdot \dot{\vec{q}}(t)+\vec{p}(t)\sdot\dot{\vec{z}}(t)-\frac{\pd\mathcal{H}}{\pd\vec{q}}\big(\vec{q}(t),\vec{p}(t)\big)\sdot\vec{z}(t) - \frac{\pd\mathcal{H}}{\pd\vec{p}}\big(\vec{q}(t),\vec{p}(t)\big) \sdot \vec{w}(t)\Big]\,\dd t.
	\]
	Osserviamo che, per le condizioni date su \(\vec{u}_\e\) e \(\vec{v}_\e\) si ha
	\[
		\vec{z}(t_1) = \vec{z}(t_2) = \vec{0} \qquad\text{e}\qquad \vec{w}(t_1)=\vec{w}(t_2) = \vec{0}.
	\]
	Integrando per parti il secondo addendo dell'integrale, otteniamo
	\[
		\int_{t_1}^{t_2}\vec{p}(t)\sdot\dot{\vec{z}}(t)\,\dd t = \underbrace{\vec{p}(t)\sdot\vec{z}(t)\bigg|_{t_1}^{t_2}}_{=0} - \int_{t_1}^{t_2} \dot{\vec{p}}(t)\sdot \vec{z}(t)\,\dd t.
	\]
	Quindi, sostituendo nell'espressione di partenza, avremo
	\[
		\int_{t_1}^{t_2}\left[\vec{z}(t)\sdot\left(-\dot{\vec{p}}(t)-\frac{\pd\mathcal{H}}{\pd\vec{q}}\big(\vec{q}(t),\vec{p}(t)\big)\right) + \vec{w}(t) \sdot \left(\dot{\vec{q}}(t) - \frac{\pd\mathcal{H}}{\pd\vec{p}}\big(\vec{q}(t),\vec{p}(t)\big)\right)\right]\,\dd t.
	\]
	Da questa espressione è chiaro che se \(\big(\vec{q}(t),\vec{p}(t)\big)\) soddisfano le equazioni di Hamilton, allora l'integrale è nullo e l'azione è resa stazionaria.
	Per mostrare il viceversa rifarsi alla dimostrazione analoga nel caso lagrangiano.
\end{proof}

\begin{defn}{Forma differenziale di Poincarè-Cartan}{formaDifferenzialePoincareCartan}
	Consideriamo l'azione sull'hamiltoniana \(\mathcal{H}=\mathcal{H}\big(\vec{q},\vec{p}\big)\), allora possiamo scrivere
	\[
		S_{t_1,t_2}^{\mathcal{H}} = \int \big(\vec{p} \sdot \dd \vec{q} - \mathcal{H}\,\dd t\big)
	\]
	che in questa forma prende il nome di \emph{forma differenziale di Poincarè-Cartan}.
\end{defn}

\begin{oss}
	La scrittura dell'integranda come forma differenziale segue da
	\[
		\int_{t_1}^{t_2} \vec{p}(t) \sdot \dot{\vec{q}}(t)\,\dd t = \int\limits_{\substack{(\vec{q}(t),\vec{p}(t))\\(\vec{q}_1,\vec{p}_1) \to (\vec{q}_2,\vec{p}_2)}} \vec{p} \sdot \dd\vec{q}.
	\]
\end{oss}

\begin{defn}{Invariante integrale di Poincarè-Cartan}{invarianteIntegralePoincareCartan}
	Si definisce \emph{invariante integrale di Poincarè-Cartan}, l'integrale della forma differenziale di Poincarè-Cartan lungo una curva chiusa \(\g\in\R^{2n}\) al tempo fissato \(t\equiv t_0\), dove si ha
	\[
		S_{t_1,t_2}^{\mathcal{H}} = \oint\limits_\g \vec{p} \sdot \dd\vec{q}.
	\]
\end{defn}

\begin{oss}
	In generale, integrando la forma differenziale lungo una curva \(\g\) qualsiasi, si ha
	\[
		\int\big(\vec{p}\sdot\dd\vec{q}-\mathcal{H}\,\dd t\big) = \int\limits_\g \vec{p}\sdot\dd\vec{q} = \int_a^b \vec{p}(\t)\sdot\frac{\pd}{\pd\t}\vec{q}(\t)\,\dd \t.
	\]
\end{oss}

\begin{prop}{Invariante di Poincarè-Cartan si conserva}{invariantePoincarèCartanConservato}
	L'invariante integrale di Poincarè-Cartan è un integrale primo del moto, per qualsiasi curva chiusa \(\g\).
\end{prop}

\begin{proof}
	Definiamo
	\[
		I_\g(0) = \oint\limits_\g \vec{p}\sdot\dd\vec{q} = \int_0^{2\p}\vec{p}(\q) \sdot \frac{\pd}{\pd\q}\vec{q}(\q)\,\dd \q,
	\]
	dove \(\q\) è una variabile angolare e \(\vec{q}(\q),\vec{p}(\q)\) sono funzioni periodiche di periodo \(2\p\). Da cui
	\[
		I_\g(t) = \oint\limits_{\j_t(\g)}\vec{p} \sdot \dd\vec{q} \qquad\text{con }\j_t(\g) = {\big(\vec{Q}_t(\q),\vec{P}_t(\q)\big)}_{\q\in[0,2\p]},
	\]
	dove \(\big(\vec{Q}_t(\q),\vec{P}_t(\q)\big)=\j_t\big(\vec{q}(\q),\vec{p}(\q)\big)\). Da cui
	\[
		I_\g(t) = \oint\limits_{\j_t(\g)}\vec{p} \sdot \dd\vec{q} = \int_0^{2\p}\vec{P}_t(\q) \sdot \frac{\pd}{\pd\q}\vec{Q}_t(\q)\,\dd\q.
	\]
	Affinché l'invariante di Poincarè-Cartan sia un integrale primo, dobbiamo dimostrare che
	\[
		\frac{\dd}{\dd t}I_\g(t) = 0.
	\]
	Ora, sfruttando le equazioni di Hamilton, otteniamo
	\[
		\begin{split}
			\frac{\dd}{\dd t}I_\g(t) & = \int_0^{2\p}\left[\dot{\vec{P}}_t(\q) \sdot \frac{\pd}{\pd\q}\vec{Q}_t(\q)+\vec{P}_t(\q) \sdot \frac{\pd}{\pd\q}\dot{\vec{Q}}_t(\q)\right]\,\dd \q\\
			& = \int_0^{2\p} \left[-\frac{\pd\mathcal{H}}{\pd\vec{q}}\big(\vec{Q}_t(\q),\vec{P}_t(\q)\big) \sdot \frac{\pd}{\pd\q}\vec{Q}_t(\q)+\vec{P}_t(\q) \sdot \frac{\pd}{\pd\q}\frac{\pd\mathcal{H}}{\pd\vec{p}}\big(\vec{Q}_t(\q),\vec{P}_t(\q)\big)\right]\,\dd \q.
		\end{split}
	\]
	Integriamo per parti il secondo addendo dell'integranda:
	\[
		\begin{split}
			\int_0^{2\p} \vec{P}_t(\q) \sdot \frac{\pd}{\pd\q}\frac{\pd\mathcal{H}}{\pd\vec{p}}\big(\vec{Q}_t(\q),\vec{P}_t(\q)\big)\,\dd \q & = \vec{P}_t(\q) \sdot \underbrace{\frac{\pd\mathcal{H}}{\pd\vec{p}}\big(\vec{Q}_t(\q),\vec{P}_t(\q)\big) \bigg|_0^{2\p}}_{=0}\\
			& - \int_0^{2\p} \frac{\pd}{\pd\q}\vec{P}_t(\q) \sdot \frac{\pd\mathcal{H}}{\pd\vec{q}}\big(\vec{Q}_t(\q),\vec{P}_t(\q)\big)\,\dd \q.
		\end{split}
	\]
	Da cui
	\[
		\frac{\dd}{\dd t}I_\g(t) = -\int_0^{2\p}\left[\frac{\dd}{\dd\q}\mathcal{H}\big(\vec{Q}_t(\q),\vec{P}_t(\q)\big)\right]\,\dd \vec{\q} = -H\big(\vec{Q}_t(\q),\vec{P}_t(\q)\big)\bigg|_0^{2\p} = 0.\qedhere
	\]
\end{proof}
%%%%%%%%%%%%%%%%%%%%%%%%%%%%%%%%%%%%%%%%%%%%%%
%
%LEZIONE 23/05/2017 - DODICESIMA SETTIMANA (1)
%
%%%%%%%%%%%%%%%%%%%%%%%%%%%%%%%%%%%%%%%%%%%%%%
\section{Trasformazioni canoniche e funzioni generatrici}

Supponiamo di avere una trasformazione \(\big(\vec{q},\vec{p}\big) \longleftrightarrow \big(\vec{Q},\vec{P}\big)\) invertibile definita dalle leggi
\[
	\begin{cases}
		\vec{Q} = \vec{Q}\big(\vec{q},\vec{p}\big) \\
		\vec{P} = \vec{P}\big(\vec{q},\vec{p}\big)
	\end{cases} \iff
	\begin{cases}
		\vec{q} = \vec{q}\big(\vec{Q},\vec{P}\big) \\
		\vec{p} = \vec{p}\big(\vec{Q},\vec{P}\big)
	\end{cases}\tag{\(1\)}
\]
Ci domandiamo se soluzioni delle equazioni di Hamilton, rispetto ad un'hamiltoniana \(\mathcal{H}=\mathcal{H}\big(\vec{q},\vec{p}\big)\), vengano mappate tramite \((1)\) in soluzioni delle equazioni di Hamilton rispetto all'hamiltoniana
\[
	\tilde{\mathcal{H}}\big(\vec{Q},\vec{P}\big) = \mathcal{H}\big(\vec{q}(\vec{Q},\vec{P}),\vec{p}(\vec{Q},\vec{P})\big).
\]
La formulazione di questa domanda è naturale in quanto ciò avviene, in maniera analoga, nel contesto lagrangiano.
D'altronde la risposta a questo quesito è, in generale, negativa.
Il motivo di ciò risiede nella differente richiesta che viene fatta, per le scelta delle leggi di trasformazione, rispetto al caso lagrangiano; in quest'ultimo la trasformazione riguardava solo le posizioni, le velocità erano infatti trasformate di conseguenza. Nel caso hamiltoniano, le leggi di trasformazione riguardano sia le posizioni che i momenti \(\vec{p}\); di fatto vengono trasformate il doppio delle coordinate e ciò giustifica la mancata covarianza generale.

Noi ci occuperemo quindi del sottoinsieme di trasformazioni su cui vi è covarianza.

\begin{defn}{Trasformazioni completamente canoniche}{trasformazioniCompletamenteCanoniche}\index{Trasformazioni canoniche}
	Si dicono \emph{trasformazioni completamente canoniche}, le trasformazione del tipo \((1)\) che preservano le equazioni di Hamilton per la generica hamiltoniana \(\mathcal{H}\big(\vec{q},\vec{p}\big)\).
\end{defn}

\begin{notz}
	Da questo momento faremo riferimento a tali trasformazioni con il solo aggettivo canoniche, pur facendo sempre riferimento a trasformazioni completamente canoniche.
\end{notz}

\begin{defn}{Funzione generatrice}{funzioneGeneratrice}\index{Funzione generatrice}
	Consideriamo una legge di trasformazione del tipo \((1)\). Si definisce \emph{funzione generatrice} una mappa \(F\) tale che
	\[
		\vec{p} \sdot \dd\vec{q} - \vec{P} \sdot \dd\vec{Q} = \dd F.
	\]
\end{defn}

\begin{notz}
	Le funzioni generatrici si distinguono in quattro tipologie:
	di I, II, III e IV specie.
\end{notz}

\begin{oss}
	La condizione che definisce le funzioni generatrici sarà chiarita dalla prossima proposizione.
\end{oss}

\begin{prop}{Condizione sufficiente per le trasformazioni canoniche}{condizioneSufficienteTrasformazioniCanoniche}
	Sia \(F\) una funzione generatrice. Supponiamo di avere una legge di trasformazione del tipo \((1)\) che soddisfa la condizione di \(F\), diremo che è generata da \(F\). Allora tale legge di trasformazione è canonica.
\end{prop}

\begin{proof}
	Un modo per garantire che le equazioni di Hamilton vengano preservate dalle trasformazioni \((1)\) è che valga
	\[
		S_{t_1,t_2}^{\mathcal{H}}\big[\big(\vec{q}(t),\vec{p}(t)\big)\big] = S_{t_1,t_2}^{\tilde{\mathcal{H}}}\big[\big(\vec{Q}(t),\vec{P}(t)\big)\big] + C \qquad\text{con }\begin{matrix}\vec{Q}(t) = \vec{Q}\big(\vec{q}(t),\vec{p}(t)\big) \\[0.5em]
			\vec{P}(t) = \vec{P}\big(\vec{q}(t),\vec{p}(t)\big)\end{matrix},\tag{\(2\)}
	\]
	dove \(C\) è una costante che, in generale, dipende dagli estremi \(\big(\vec{q}_1,\vec{p}_1\big),\big(\vec{q}_2,\vec{p}_2\big)\) ma non dalla traiettoria della curva. Osserviamo che, esplicitamente, la condizione \((2)\) equivale a
	\[
		\int\limits_\Gamma \big(\vec{p}\sdot \dd\vec{q} - \mathcal{H}\,\dd t\big) = \int\limits_{\tilde{\Gamma}} \big(\vec{P} \sdot \dd\vec{Q} - \tilde{\mathcal{H}}\,\dd t\big) + \int\limits_{\tilde{\Gamma}}\dd F,
	\]
	dove \(C\) ha assunto la forma dell'integrale di un differenziale esatto, che rispetta le condizioni che avevamo richiesto.
	Pertanto, un modo per garantire che la \((2)\) valga, per ogni coppia di \((\mathcal{H} \longleftrightarrow \tilde{\mathcal{H}})\), è di chiedere che
	\[
		\vec{p} \sdot \dd\vec{q} - \mathcal{H}\,\dd t = \vec{P} \sdot \dd\vec{Q} - \tilde{\mathcal{H}}\,\dd t + \dd F.
	\]
	Per costruzione \(\mathcal{H}=\tilde{\mathcal{H}}\), in quanto
	\[
		\tilde{\mathcal{H}}\big(\vec{Q},\vec{P}\big) = \mathcal{H}\big(\vec{q}(\vec{Q},\vec{P}),\vec{p}(\vec{Q},\vec{P})\big).
	\]
	Quindi la condizione precedente si riscrive come
	\[
		\vec{p} \sdot \dd\vec{q} = \vec{P} \sdot \dd\vec{Q} + \dd F \iff \vec{p}\sdot \dd\vec{q} - \vec{P}\sdot \dd\vec{Q} = \dd F.
	\]
	Ovvero la legge di trasformazione è generata da \(F\). 
\end{proof}

Procediamo ora con una descrizione delle quattro tipologie di funzioni generatrici.
Osserveremo che, una volta definita la prima, le altre sfruttano la medesima logica.

\begin{defn}{Funzione generatrice di prima specie}{funzioneGeneratricePrimaSpecie}
	\(F=F\big(\vec{q},\vec{Q}\big)\) si definisce \emph{funzione generatrice di prima specie} se soddisfa le seguenti condizioni
	\[
		\begin{cases}
			\vec{p} = \frac{\pd F}{\pd\vec{q}}\big(\vec{q},\vec{Q}\big) \\
			\vec{P} = -\frac{\pd F}{\pd\vec{Q}}\big(\vec{q},\vec{Q}\big)
		\end{cases}
	\]
	e se la prima di tali condizioni è invertibile rispetto a \(\vec{Q}\).
\end{defn}

\begin{pr}
	Le funzioni generatrici di prima specie sono funzioni generatrici di trasformazioni canoniche.
\end{pr}

\begin{proof}
	Consideriamo la mappa \(F=F\big(\vec{q},\vec{Q}\big)\), avremo
	\[
		\dd F = \frac{\pd F}{\pd\vec{q}}\sdot \dd\vec{q} + \frac{\pd F}{\pd\vec{Q}} \sdot \dd\vec{Q}.
	\]
	Quindi la condizione affinché \(F\) sia una funzione generatrice è soddisfatta se e solo se vengono soddisfatte le condizioni di definizione di \(F\), infatti
	\[
		\vec{p} \sdot \dd\vec{q} - \vec{P} \sdot \dd\vec{Q} = \dd F \iff 	\begin{cases}
			\vec{p} = \frac{\pd F}{\pd\vec{q}}\big(\vec{q},\vec{Q}\big) \\
			\vec{P} = -\frac{\pd F}{\pd\vec{Q}}\big(\vec{q},\vec{Q}\big)
		\end{cases}
	\]
	Supponiamo ora che la prima relazione si invertibile rispetto a \(\vec{Q}\). Pertanto avremo \(\vec{Q}=\vec{Q}\big(\vec{q},\vec{p}\big)\).
	Sostituendo tale trasformazione nella seconda relazione si ottiene
	\[
		\vec{P} = -\frac{\pd F}{\pd\vec{Q}}\big(\vec{q},\vec{Q}\big) = -\frac{\pd F}{\pd\vec{Q}}\big(\vec{q},\vec{Q}(\vec{q},\vec{p})\big) \equiv \vec{P}\big(\vec{q},\vec{p}\big).
	\]
	Se tale procedura ci fornisce una relazione del tipo \((1)\), allora tale costruzione è automaticamente canonica.
\end{proof}

\begin{ese}[Scambio delle coordinate]
	Consideriamo la funzione generatrice di I specie 
	\[
		F\big(\vec{q},\vec{Q}\big) = \vec{q} \sdot \vec{Q}.
	\]
	Imponendo le condizioni si ottiene
	\[
		\begin{cases}
			\vec{p} = \frac{\pd}{\pd\vec{q}} \big(\vec{q} \sdot \vec{Q}\big) = \vec{Q} \\
			\vec{P} = -\frac{\pd}{\pd\vec{Q}} \big(\vec{q} \sdot \vec{Q}\big) = \vec{q}
		\end{cases} \iff
		\begin{cases}
			\vec{Q} = \vec{p} \\
			\vec{P} = -\vec{q}
		\end{cases}
	\]
	Quindi, tramite una trasformazione canonica, abbiamo ottenuto una legge in cui il ruolo di momenti e posizioni sono scambiati. Ciò nonostante la struttura e la validità del moto è mantenuta inalterata.
\end{ese}

\begin{defn}{Funzione generatrice si seconda specie}{funzioneGeneraticieSecondaSpecie}
	\(G=G\big(\vec{q},\vec{P}\big)\) so definisce \emph{funzione generatrice di seconda specie} se soddisfa le seguenti condizioni
	\[
		\begin{cases}
			\vec{p} = \frac{\pd G}{\pd\vec{q}}\big(\vec{q},\vec{P}\big) \\
			\vec{Q} = \frac{\pd G}{\pd\vec{p}}\big(\vec{q},\vec{P}\big)
		\end{cases}
	\]
	e se la prima di tali condizioni è invertibile rispetto a \(\vec{P}\).
\end{defn}

\begin{pr}
	Le funzioni generatrici di seconda specie sono funzioni generatrici di trasformazioni canoniche.
\end{pr}

\begin{proof}
	Scrivendo
	\[
		\vec{P} \sdot \dd\vec{Q} = \dd\big(\vec{P}\sdot\vec{Q})-\vec{Q} \sdot \dd\vec{P},
	\]
	e sostituendo nella condizione delle funzioni generatrici, otteniamo
	\[
		\vec{p} \sdot \dd\vec{q} - \vec{P} \sdot \dd\vec{Q} = \dd F \iff \vec{p} \sdot \dd\vec{q} + \vec{Q} \sdot \dd\vec{P} = \dd\big(F-\vec{P} \sdot \vec{Q}\big) \equiv \dd G,
	\]
	dove \(G=G\big(\vec{q},\vec{P}\big)\).
	A questo punto è sufficiente applicare la dimostrazione per le funzioni generatrici di prima specie a \(G\) per ottenere la tesi.
\end{proof}

\begin{ese}[Funzione generatrice della trasformazione identica]
	La trasformazione identica è generata dalla trasformazione di II specie
	\[
		G\big(\vec{q},\vec{P}\big) = \vec{q} \sdot \vec{P},
	\]
	infatti si ottiene
	\[
		\begin{cases}
			\vec{p} = \frac{\pd}{\pd\vec{q}}\big(\vec{q}\sdot\vec{P}\big) = \vec{P} \\
			\vec{Q} = \frac{\pd}{\pd\vec{P}}\big(\vec{q}\sdot\vec{P}\big) = \vec{q}
		\end{cases} \iff
		\begin{cases}
			\vec{Q} = \vec{q} \\
			\vec{P} = \vec{p}
		\end{cases}
	\]
\end{ese}

\begin{defn}{Funzione generatrice si terza specie}{funzioneGeneraticieTerzaSpecie}
	\(\y=\y\big(\vec{p},\vec{Q}\big)\) so definisce \emph{funzione generatrice di terza specie} se soddisfa le seguenti condizioni
	\[
		\begin{cases}
			\vec{q} = \frac{\pd \y}{\pd\vec{p}}\big(\vec{p},\vec{Q}\big) \\
			\vec{P} = \frac{\pd \y}{\pd\vec{Q}}\big(\vec{p},\vec{Q}\big)
		\end{cases}
	\]
	e se la prima di tali condizioni è invertibile rispetto a \(\vec{Q}\).
\end{defn}

\begin{pr}
	Le funzioni generatrici di terza specie sono funzioni generatrici di trasformazioni canoniche.
\end{pr}

\begin{proof}
	Scrivendo
	\[
		\vec{p} \sdot \dd\vec{q} = \dd\big(\vec{p}\sdot\vec{q}\big) - \vec{q}\sdot \dd\vec{p},
	\]
	e sostituendo nella condizione delle funzioni generatrici, otteniamo
	\[
		\vec{p} \sdot \dd\vec{q} - \vec{P} \sdot \dd\vec{Q} = \dd F \iff \vec{q} \sdot \dd\vec{p} + \vec{P} \sdot \dd\vec{Q} = \dd\big(-F+\vec{p}\sdot\vec{q}\big) \equiv \dd\y
	\]
	dove \(\y=\y\big(\vec{p},\vec{Q}\big)\).
	A questo punto è sufficiente applicare la dimostrazione per le funzioni generatrici di prima specie a \(\y\) per ottenere la tesi.
\end{proof}

\begin{defn}{Funzione generatrice si quarta specie}{funzioneGeneraticieQuartaSpecie}
	\(\phi=\phi\big(\vec{p},\vec{P}\big)\) so definisce \emph{funzione generatrice di quarta specie} se soddisfa le seguenti condizioni
	\[
		\begin{cases}
			\vec{q} = \frac{\pd \phi}{\pd\vec{p}}\big(\vec{p},\vec{P}\big) \\
			\vec{Q} = -\frac{\pd \phi}{\pd\vec{P}}\big(\vec{p},\vec{P}\big)
		\end{cases}
	\]
	e se la prima di tali condizioni è invertibile rispetto a \(\vec{P}\).
\end{defn}

\begin{pr}
	Le funzioni generatrici di quarta specie sono funzioni generatrici di trasformazioni canoniche.
\end{pr}

\begin{proof}
	Scrivendo
	\[
		\vec{p}\sdot \dd\vec{q} = \dd\big(\vec{p}\sdot \vec{q}\big) - \vec{q}\sdot\dd\vec{p} \qquad\text{e}\qquad \vec{P} \sdot \dd\vec{Q} = \dd\big(\vec{P}\sdot\vec{Q})-\vec{Q} \sdot \dd\vec{P},
	\]
	e sostituendo nella condizione delle funzioni generatrici, otteniamo
	\[
		\vec{p} \sdot \dd\vec{q} - \vec{P} \sdot \dd\vec{Q} = \dd F \iff \vec{q}\sdot \dd\vec{p} - \vec{Q}\sdot \dd\vec{P} = \dd\big(-F+\vec{p}\sdot\vec{q} - \vec{P}\sdot\vec{Q}\big) \equiv \dd\phi,
	\]
	dove \(\phi=\phi\big(\vec{p},\vec{P}\big)\).
	A questo punto è sufficiente applicare la dimostrazione per le funzioni generatrici di prima specie a \(\phi\) per ottenere la tesi.
\end{proof}

\section{Equazione di Hamilton-Jacobi}

Immaginiamo di avere una legge di trasformazione della posizione che riguardi solo la posizione, ovvero \(\vec{Q}=\vec{Q}\big(\vec{q}\big)\).
Per ottenere una trasformazione canonica tale che la trasformazione delle posizioni abbia questa forma, possiamo generarla tramite la funzione generatrice di seconda specie
\[
	G\big(\vec{q},\vec{P}\big) = \vec{Q}\big(\vec{q}\big) \sdot \vec{P}.
\]
Infatti, imponendo le condizioni derivanti da \(G\), otteniamo
\[
	\begin{cases}
		\vec{p} = \frac{\pd\utilde{Q}}{\pd\vec{q}} \, \utilde{P} \\
		\vec{Q} = \vec{Q}\big(\vec{q}\big),
	\end{cases}
\]
dove la prima relazione può essere letta in coordinate come
\[
	p_i = \sum_{j=1}^n \frac{\pd Q_j}{\pd q_i}\,P_j \equiv {\big(M\,\vec{P}\big)}_i \qquad\text{dove }M_{ij}\big(\vec{q}\big) = \frac{\pd Q_j}{\pd q_i}\big(\vec{q}\big)
\]
è la matrice jacobiana della trasformazione \(\vec{Q}\big(\vec{q}\big)\). Da cui otteniamo la seguente definizione

\begin{defn}{Trasformazioni canoniche puntuali}{trasformazioniCanonichePuntuali}\index{Trasformazioni canoniche!puntuali}
	Una trasformazione canoniche si dice \emph{puntuale} se è della forma
	\[
		\begin{cases}
			\vec{Q} = \vec{Q}\big(\vec{q}\big) \\
			\vec{P} = M^{-1}\big(\vec{q}\big)\,\vec{p},
		\end{cases}
	\]
	dove \(M\big(\vec{q}\big)\) è la matrice jacobiana della trasformazione \(\vec{Q}\big(\vec{q}\big)\).
\end{defn}

\begin{oss}
	Nel contesto lagrangiano queste trasformazioni, che riguardavano solo la posizione, erano accompagnate dalle trasformazioni della velocità da esse indotte:
	\[
		\dot{\vec{Q}} = \frac{\pd\vec{Q}}{\pd\utilde{q}}\sdot \dot{\utilde{q}}.
	\]
\end{oss}

L'interesse pratico per trasformazioni che preservano le equazioni di Hamilton riguarda la risoluzione di queste ultime.
L'idea, assegnate delle equazioni di Hamilton in generali difficili da risolvere, è quella di cercare nuove coordinate, legate alle precedenti da una trasformazione canonica, sulle quali le equazioni di Hamilton siano banali da risolvere.

Un caso in cui ciò accade è quello dove \(\tilde{\mathcal{H}}\big(\vec{Q},\vec{P}\big)=\tilde{\mathcal{H}}\big(\vec{P}\big)\) è indipendente dalle \(\vec{Q}\). In tal caso, per le equazioni di Hamilton, si ha
\[
	\dot{\vec{P}} = -\frac{\pd\tilde{\mathcal{H}}}{\pd\vec{Q}} = 0 \implies \vec{P}(t) = \vec{P}_0
\]
e, di conseguenza,
\[
	\dot{\vec{Q}} = \frac{\pd\tilde{\mathcal{H}}}{\pd\vec{P}}\big(\vec{P}_0\big) = \vec{V}\big(\vec{P}_0\big) \implies \vec{Q}(t) = \vec{Q}_0 + V\big(\vec{P}_0\big)\,t.
\]
Cerchiamo quindi una trasformazione canonica generata da una funzione generatrice di II specie \(G\big(\vec{q},\vec{P}\big)\) che abbia che proprietà. In particolare vorremmo che, tramite le condizioni
\[
	\begin{cases}
		\vec{p} = \frac{\pd G}{\pd\vec{q}}\big(\vec{q},\vec{P}\big) \\
		\vec{Q} = \frac{\pd G}{\pd\vec{P}}\big(\vec{q},\vec{P}\big)
	\end{cases}
\]
si abbia \(\mathcal{H}\big(\vec{q}(\vec{Q},\vec{P}),\vec{p}(\vec{Q},\vec{P})\big) = \tilde{\mathcal{H}}\big(\vec{P}\big)\).
Sostituendo \(\vec{p}\) tramite le condizioni della funzione generatrice, si ottiene

\begin{remark}{Equazione di Hamilton-Jacobi}{equazioneHamiltonJacobi}\index{Equazione di Hamilton-Jacobi}
	\[
		\mathcal{H}\left(\vec{q}\big(\vec{Q},\vec{P}\big), \frac{\pd G}{\pd\vec{q}}\big(\vec{q}(\vec{Q},\vec{P}),\vec{P}\big)\right) = \tilde{\mathcal{H}}\big(\vec{P}\big)
	\]
\end{remark}

\begin{oss}
	La possibilità di sfruttare le equazioni di Hamilton-Jacobi per risolvere in modo sistematico le equazioni di Hamilton, giustifica il formalismo Hamiltoniano. Tale possibilità non esiste infatti nel contesto lagrangiano.
	Ciò nonostante non è detto che l'equazione di Hamilton-Jacobi ammetta sempre soluzione.
\end{oss}

\begin{ese}
	Consideriamo la lagrangiana
	\[
		\mathcal{L}(q,\dot{q}) = \frac{1}{4}q^2 \dot{q}^2 e^{q^2}.
	\]
	Ricaviamo l'hamiltoniana associata ad \(\mathcal{L}\) e scriviamo le equazioni di Hamilton corrispondenti.
	Per definizione
	\[
		\mathcal{H}(q,p) = p\,\dot{q} - \mathcal{L}(q,\dot{q}),
	\]
	dove
	\[
		p = \frac{\pd\mathcal{L}}{\pd \dot{q}} = \frac{1}{2}q^2 \dot{q}\,e^{q^2} \iff \dot{q} = \frac{2p}{q^2}e^{-q^2},
	\]
	con \(p\) che è invertibile rispetto a \(\dot{q}\) per \(q\in (0,+\infty)\) oppure \(q\in (-\infty,0)\).
	Scegliamo la prima determinazione, segue
	\[
		\mathcal{H}(q,p) = \frac{2p^2}{q^2}e^{-q^2} -\frac{1}{4}q^2 \frac{4p^2}{q^4}\,e^{-2q^2} e^{q^2} = \frac{p^2}{q^2}\,e^{-q^2}.
	\]
	Da cui le equazioni di Hamilton
	\[
		\begin{cases}
			\dot{q} = \frac{2p}{q^2}\,e^{-q^2} \\
			\dot{p} = \frac{2p^2}{q}\,e^{-q^2} \left(1+\frac{1}{q^2}\right)
		\end{cases}
	\]
	Risolviamo ora le equazioni di Hamilton tramite coordinate opportune, queste ultime trovate sfruttando le equazioni di Hamilton-Jacobi.
	Scegliamo arbitrariamente, in modo da semplificarci i calcoli, \(\tilde{\mathcal{H}}(P)=P^2\). Applicando l'equazione di Hamilton-Jacobi avremo
	\[
		\mathcal{H}\left(q,\frac{\pd G}{\pd q}\right) = \tilde{\mathcal{H}}(P) = P^2 \iff \frac{1}{q^2}e^{-q^2} \left( \frac{\pd G}{\pd q} \right)^2 = P^2,
	\]
	da cui
	\[
		\left( \frac{\pd G}{\pd q} \right)^2 = q^2 p^2 e^{q^2} \implies \frac{\pd G}{\pd q} = \pm q\,P\,e^{\frac{q^2}{2}}.
	\]
	Scegliamo la determinazione positiva della radice per ottenere
	\[
		\frac{\pd G}{\pd q} = q\,P\,e^{\frac{q^2}{2}} \implies G(q,P) = P\,e^{\frac{q^2}{2}}.
	\]
	Con tale scelta, imponendo le condizioni delle funzioni generatrici del II tipo, otteniamo
	\[
		\begin{cases}
			p = \frac{\pd G}{\pd q}(q,P) = q\,P\,e^{\frac{q^2}{2}} \\
			Q = \frac{\pd G}{\pd P}(q,P) = e^{\frac{q^2}{2}}
		\end{cases}\iff
		\begin{cases}
			Q = e^{\frac{q^2}{2}} \\
			P = \frac{p}{q}\,e^{-\frac{q^2}{2}}
		\end{cases}
	\]
	che, come ci aspettavamo, è una trasformazione puntuale. Osserviamo inoltre che
	\[
		(q,p) \in (0,+\infty) \times \R \longleftrightarrow (Q,P) \in (1,+\infty) \times \R.
	\]
	La trasformazione inversa si ottiene facilmente:
	\[
		\begin{cases}
			q = \sqrt{2\ln Q} \\
			p = P\,Q\,\sqrt{2\ln Q}
		\end{cases}
	\]
	Nelle nuove coordinate, per costruzione \(\tilde{\mathcal{H}}(P)=P^2\), da cui otteniamo le equazioni di Hamilton
	\[
		\begin{cases}
			\dot{Q} = 2P \\
			\dot{P} = 0
		\end{cases} \implies
		\begin{cases}
			Q(t) = Q_0 +2 P_0 t \\
			P(t) = P_0
		\end{cases} \qquad\text{con } Q(t) > 1 \iff t > \frac{1-Q_0}{2P_0}.
	\]
	Tornando alle coordinate iniziali otteniamo
	\[
		\begin{cases}
			q(t) = \sqrt{2\ln(Q_0+2P_0 t)} \\
			p(t) = P_0 (Q_0+2P_0 t)\,\sqrt{2\ln(Q_0+2P_0 t)}
		\end{cases}
	\]
	che risolvono il moto.
\end{ese}
%%%%%%%%%%%%%%%%%%%%%%%%%%%%%%%%%%%%%%%%%%%%%%
%
%LEZIONE 24/05/2017 - DODICESIMA SETTIMANA (2)
%
%%%%%%%%%%%%%%%%%%%%%%%%%%%%%%%%%%%%%%%%%%%%%%
\section{Caratterizzazione delle trasformazioni canoniche}

\begin{teor}{Caratterizzazione delle trasformazioni canoniche}{caratterizzazioneTrasformazioniCanoniche}
	Una trasformazione \(\big(\vec{q},\vec{p}\big) \longleftrightarrow \big(\vec{Q},\vec{P}\big)\) è canonica se e soltanto se la matrice jacobiana \(M\) della trasformazione soddisfa
	\[
		M\,J \tran{M} = J,
	\]
	dove \(J\) è la \hyperref[df:matriceSimpletticaStandard]{matrice simplettica standard}.
\end{teor}

\begin{proof}
	Siano \(\big(\vec{q}(t),\vec{p}(t)\big)\) soluzioni delle equazioni di Hamilton associate all'hamiltoniana \(\mathcal{H}\). Dalla legge di trasformazione, derivando otteniamo
	\[
		\begin{cases}
			\vec{Q}(t) = \vec{Q}\big(\vec{q}(t),\vec{p}(t)\big) \\[0.5em]
			\vec{P}(t) = \vec{P}\big(\vec{q}(t),\vec{p}(t)\big)
		\end{cases} \implies
		\begin{cases}
			\dot{\vec{Q}} = \frac{\pd\vec{Q}}{\pd\utilde{q}} \, \dot{\utilde{q}} + \frac{\pd\vec{Q}}{\pd\utilde{p}}\,\dot{\utilde{p}} \\[0.5em]
			\dot{\vec{P}} = \frac{\pd\vec{P}}{\pd\utilde{q}}\,\dot{\utilde{q}} + \frac{\vec{P}}{\utilde{p}}\,\dot{\utilde{p}}
		\end{cases}
	\]
	da cui, sostituendo le equazioni di Hamilton per \(\dot{\vec{q}}\) e \(\dot{\vec{p}}\), si ottiene
	\[
		\begin{cases}
			\dot{\vec{Q}} = \frac{\pd\vec{Q}}{\pd\utilde{q}} \, \frac{\pd\mathcal{H}}{\pd\utilde{p}} - \frac{\pd\vec{Q}}{\pd\utilde{p}}\,\frac{\pd\mathcal{H}}{\pd\utilde{q}} \\[0.5em]
			\dot{\vec{P}} = \frac{\pd\vec{P}}{\pd\utilde{q}} \, \frac{\pd\mathcal{H}}{\pd\utilde{p}} - \frac{\pd\vec{P}}{\pd\utilde{p}} \, \frac{\pd\mathcal{H}}{\pd\utilde{q}}
		\end{cases}
	\]
	Introduciamo la matrice di jacobi della trasformazione, adottando una notazione di matrice a blocchi:
	\[
		M = \begin{pmatrix}
			\frac{\pd\vec{Q}}{\pd\utilde{q}} & \frac{\pd\vec{Q}}{\pd\utilde{p}} \\[0.5em]\\
			\frac{\pd\vec{P}}{\pd\utilde{q}} & \frac{\pd\vec{P}}{\pd\utilde{p}}
		\end{pmatrix} = 
		\begin{pmatrix}
			A & B  \\
			C & D
		\end{pmatrix},
	\]
	dove \(A,B,C,D\) sono matrici \(n \times n\). Pertanto, sfruttando tale notazione nella scrittura di \(\dot{\vec{Q}}\) e \(\dot{\vec{P}}\), avremo
	\[
		\begin{cases}
			\dot{Q}_i = \sum_j \left(A_{ij} \frac{\pd\mathcal{H}}{\pd p_j} - B_{ij} \frac{\pd\mathcal{H}}{\pd q_j}\right) \\[0.5em]
			\dot{P}_i = \sum_j \left(C_{ij} \frac{\pd\mathcal{H}}{\pd p_j} - D_{ij} \frac{\pd\mathcal{H}}{\pd q_j}\right)
		\end{cases}\implies
		\begin{cases}
			\dot{\vec{Q}} = A\,\frac{\pd\mathcal{H}}{\pd\vec{p}} - B\,\frac{\pd\mathcal{H}}{\pd\vec{q}} \\[0.5em]
			\dot{\vec{P}} = C\,\frac{\pd\mathcal{H}}{\pd\vec{p}} - D\,\frac{\pd\mathcal{H}}{\pd\vec{q}}
		\end{cases}
	\]
	ovvero
	\[
		\begin{pmatrix}\dot{\vec{Q}}\\[0.5em] \dot{\vec{P}}\end{pmatrix} = \begin{pmatrix}A & B\\[0.5em]C & D\end{pmatrix} \begin{pmatrix}
			\frac{\pd\mathcal{H}}{\pd\vec{p}} \\[0.5em]
			-\frac{\pd\mathcal{H}}{\pd\vec{q}}
		\end{pmatrix}
	\]
	Adottiamo la seguente notazione per il resto della dimostrazione:
	\[
		\vec{x} = \begin{pmatrix}\vec{q}\\[0.5em] \vec{p}\end{pmatrix} \qquad\text{e}\qquad \vec{X} = \begin{pmatrix}\vec{Q}\\[0.5em] \vec{P}\end{pmatrix}
	\]
	Ricordiamo che il \hyperref[df:campoVettorialeHamiltoniano]{campo vettoriale hamiltoniano}, può essere riscritto tramite la matrice simplettica standard \(J\):
	\[
		\begin{pmatrix}
			\frac{\pd\mathcal{H}}{\pd\vec{p}} \\[0.5em]
			-\frac{\pd\mathcal{H}}{\pd\vec{q}}
		\end{pmatrix} =
		\begin{pmatrix}
			0  & -Id \\[0.5em]
			Id & 0
		\end{pmatrix}
		\begin{pmatrix}
			\frac{\pd\mathcal{H}}{\pd\vec{q}} \\[0.5em]
			\frac{\pd\mathcal{H}}{\pd\vec{p}}
		\end{pmatrix} \equiv J\,\frac{\pd\mathcal{H}}{\pd\vec{x}}
	\]
	In questi termini, le equazioni del moto hanno la seguente forma sintetica
	\[
		\dot{\vec{X}} = M\,J\,\frac{\pd\mathcal{H}}{\pd\vec{x}}
	\]
	A questo punto vogliamo imporre la seguente condizione
	\[
		M\,J\,\frac{\pd\mathcal{H}}{\pd\vec{x}} = J\,\frac{\pd\tilde{\mathcal{H}}}{\pd\vec{X}}.
	\]
	Da \(\mathcal{H}\big(\vec{q},\vec{p}\big) = \tilde{\mathcal{H}}\big(\vec{Q}(\vec{q},\vec{p}),\vec{P}(\vec{q},\vec{p})\big)\), otteniamo
	\[
		\begin{cases}
			\frac{\pd\mathcal{H}}{\pd\vec{q}} = \frac{\pd\tilde{\mathcal{H}}}{\pd\utilde{Q}}\,\frac{\pd\utilde{Q}}{\pd\vec{q}} + \frac{\pd\tilde{\mathcal{H}}}{\pd\utilde{P}}\,\frac{\utilde{P}}{\pd\vec{q}} \\[0.5em]
			\frac{\pd\mathcal{H}}{\pd\vec{p}} = \frac{\pd\tilde{\mathcal{H}}}{\pd\utilde{Q}}\,\frac{\pd\utilde{Q}}{\pd\vec{p}} + \frac{\pd\tilde{\mathcal{H}}}{\pd\utilde{P}}\,\frac{\utilde{P}}{\pd\vec{p}}
		\end{cases}
	\]
	Sfruttando la definizione di \(M\) data in precedenza, avremo
	\[
		\frac{\pd\mathcal{H}}{\pd q_i} = \sum_j A_{ji} \frac{\pd\tilde{\mathcal{H}}}{\pd Q_j} + C_{ji} \frac{\pd\tilde{\mathcal{H}}}{\pd P_j} = \sum_{j}{(\tran{A})}_{ij} \frac{\pd\tilde{\mathcal{H}}}{\pd Q_j} + {(\tran{C})}_{ij} \frac{\pd\tilde{\mathcal{H}}}{\pd P_j} \implies \frac{\pd\mathcal{H}}{\pd\vec{q}} = \tran{A}\,\frac{\pd\tilde{\mathcal{H}}}{\pd\vec{Q}} + \tran{C}\,\frac{\pd\tilde{\mathcal{H}}}{\pd\vec{P}}
	\]
	e, analogamente,
	\[
		\frac{\pd\mathcal{H}}{\pd\vec{p}} = \tran{B}\,\frac{\pd\tilde{\mathcal{H}}}{\pd\vec{Q}} + \tran{D}\,\frac{\pd\tilde{\mathcal{H}}}{\pd\vec{P}}.
	\]
	Tali scritture, in forma compatta, diventano
	\[
		\frac{\pd\mathcal{H}}{\pd\vec{x}} = \begin{pmatrix}
			\tran{A} & \tran{C} \\[0.5em]
			\tran{B} & \tran{D}
		\end{pmatrix}
		\begin{pmatrix}
			\frac{\pd\tilde{\mathcal{H}}}{\pd\vec{Q}} \\[0.5em]
			\frac{\pd\tilde{\mathcal{H}}}{\pd\vec{P}}
		\end{pmatrix} = \tran{M}\,\frac{\pd\tilde{\mathcal{H}}}{\pd\vec{X}}
	\]
	che, sostituita nell'equazione del moto, ci permette di scrivere
	\[
		\dot{\vec{X}} = M\,J\,\frac{\pd\mathcal{H}}{\pd\vec{x}} = M\,J \tran{M}\,\frac{\pd\tilde{\mathcal{H}}}{\pd\vec{X}}.
	\]
	La quale soddisfa la condizione imposta in precedenza se e soltanto se
	\[
		M\,J \tran{M} = J.\qedhere
	\]
\end{proof}

\begin{defn}{Matrice simplettica}{matriceSimplettica}\index{Matrice simplettica}
	Una matrice \(M\), di dimensioni \(2n \times 2n\), si dice \emph{simplettica} se soddisfa
	\[
		M\,J \tran{M} = J,
	\]
	dove \(J\) è la matrice simplettica standard.
\end{defn}

\begin{oss}
	La matrice simplettica standard è una matrice simplettica, infatti se prendiamo \(M=J\), avremo
	\[
		M\,J \tran{M} = J\,J \tran{J} = J^2 \tran{J}.
	\]
	Ora
	\[
		J^2 = 	\begin{pmatrix}0 & -Id\\Id & 0\end{pmatrix}\begin{pmatrix}0 & -Id\\Id & 0\end{pmatrix} = \begin{pmatrix}-Id & 0\\0 & -Id\end{pmatrix} = -Id_{2n}
	\]
	da cui
	\[
		J^2 \tran{J} = -\tran{J} = J
	\]
	in quanto \(J\) è antisimmetrica.
\end{oss}

\begin{pr}
	Sia \(M\) una matrice simplettica. Allora
	\[
		\det M = 1
	\]
\end{pr}

\begin{proof}
	Dimostriamo con un argomento semplice che \(\det M = \pm 1\), per dimostrare completamente la tesi è necessario qualche sforzo ulteriore.
	Calcoliamo inizialmente il determinante di \(J\):
	\[
		\det J = \det \begin{pmatrix}0 & -Id\\Id & 0\end{pmatrix} = (-1)^n \det \begin{pmatrix}Id & 0\\0 & -Id\end{pmatrix} = (-1)^n (-1)^n = 1 \,\fa n.
	\]
	A questo punto, per calcolare il determinante di \(M\), sfruttiamo la condizione delle matrici simplettiche
	\[
		\det(M\,J \tran{M}) = \det J \iff \det M\,\det \tran{M} = 1 \iff {(\det M)}^2 = 1,
	\]
	da cui
	\[
		\det M = \pm 1. \qedhere
	\]
\end{proof}

\begin{pr}
	Sia \(M\) una matrice simplettica. Allora
	\[
		M^{-1} = -J \tran{M}\, J.
	\]
\end{pr}

\begin{proof}
	Dalla definizione di matrice simplettica \(M\,J \tran{M} = J\), da cui
	\[
		J \tran{M} = M^{-1} J \implies J \tran{M}\,J^{-1} = M^{-1},
	\]
	ovvero, dal momento che \(J\) è antisimmetrica,
	\[
		M^{-1} = -J \tran{M}\, J.\qedhere
	\]
\end{proof}

\begin{prop}{Conservazione delle parentesi di Poisson fondamentali}{conservazioneParentesiPoissonFond}
	Una trasformazione \(\big(\vec{q},\vec{p}\big) \longleftrightarrow \big(\vec{Q},\vec{P}\big)\) è canonica se e soltanto se preserva le parentesi di Poisson fondamentali, ovvero
	\[
		\{Q_i,Q_j\} = 0; \qquad \{P_i,P_j\} = 0; \qquad \{Q_i,P_j\} = \d_{i,j}.
	\]
\end{prop}

\begin{proof}
	Detta \(M\) la matrice jacobiana della trasformazione, sappiamo che la trasformazione è canonica se e soltanto se 
	\[
		M\,J \tran{M} = J,
	\]
	con \(J\) la matrice simplettica standard. Tale condizione può essere riscritta come segue:
	\[
		\begin{split}
			M\,J\tran{M} & = \begin{pmatrix}A & B\\C & D\end{pmatrix} \begin{pmatrix}0 & -Id\\Id & 0\end{pmatrix} \begin{pmatrix}\tran{A} & \tran{C}\\\tran{B} & \tran{D}\end{pmatrix} = \begin{pmatrix}A & B\\C & D\end{pmatrix} \begin{pmatrix}-\tran{B} & -\tran{D}\\\tran{A} & \tran{C}\end{pmatrix}\\
			& = \begin{pmatrix}
				-A\tran{B} + B\tran{A} & -A\tran{D} + B\tran{C} \\
				-C\tran{B} + D\tran{A} & -C\tran{D} + D\tran{C}
			\end{pmatrix} = \begin{pmatrix}0 & -Id\\Id & 0\end{pmatrix} = J.
		\end{split}
	\]
	Da cui si ottengono tre condizioni:
	\[
		A\tran{B} - B\tran{A} = 0; \qquad C\tran{D} - D\tran{C} = 0; \qquad A\tran{D} - B\tran{C} = Id.
	\]
	Ricordando l'espressione di \(A,B\), la prima condizione diventa
	\[
		\sum_k A_{ik} {(\tran{B})}_{kj} - B_{ik} {(\tran{A})}_{kj} = 0 \iff \sum_k (A_{ik} B_{jk} - B_{ik} A_{jk}) = 0,
	\]
	da cui
	\[
		\sum_k \bigg(\frac{\pd Q_i}{\pd q_k}\,\frac{\pd Q_j}{\pd p_k} - \frac{\pd Q_i}{\pd p_k}\,\frac{\pd Q_i}{\pd q_k}\bigg) = 0 \iff \{Q_i,Q_j\} = 0.
	\]
	Analogamente la seconda condizione ci dice
	\[
		C\tran{D} - D\tran{C} = 0 \iff \{P_i,P_j\} = 0.
	\]
	Infine, per la terza condizione, si ha
	\[
		A\tran{D} - B\tran{C} = Id \iff \sum_k (A_{ik} D_{jk} - B_{ik} C_{jk}) = \d_{ij},
	\]
	da cui
	\[
		\sum_k \bigg(\frac{\pd Q_i}{\pd q_k}\,\frac{\pd P_j}{\pd p_k} - \frac{\pd Q_i}{\pd p_k}\,\frac{\pd P_j}{\pd q_k}\bigg) = \d_{i,j} \iff \{Q_i,P_j\} = \d_{i,j}.
	\]
	Pertanto abbiamo dimostrato che la trasformazione è canonica se e soltanto se sono preservate le parentesi di Poisson fondamentali.
\end{proof}