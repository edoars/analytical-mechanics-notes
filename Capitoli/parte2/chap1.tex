%!TEX root = ../../main.tex
%%%%%%%%%%%%%%%%%%%%%%%%%%%%%%%%%%%%%%%%%
%
%LEZIONE 13/03/2017 - TERZA SETTIMANA (1)
%
%%%%%%%%%%%%%%%%%%%%%%%%%%%%%%%%%%%%%%%%%
\chapter{Sistemi conservativi unidimensionali}

L'equazione generale del moto, in un sistema conservativo unidimensionale, è data da
\[
	m\,\ddot{x}(t) = -U'\big(x(t)\big).
\]

\begin{oss}\label{oss:conservazioneUnidim}
	Come al solito, quando si studia un sistema conservativo, è naturale domandarsi quando, in presenza di un'equazione di forze posizionali
	\[
		m\,\ddot{x} = F(x),
	\]
	sia lecito scrivere \(F(x) = -U'(x)\).
	
	Nel caso monodimensionale, ogni forza posizionale è conservativa in quanto continua. Infatti per il Teorema Fondamentale del Calcolo si ha
	\[
		U(x) = -\int_0^x F(t)\,\dd t.
	\]
\end{oss}

\section{Pendolo matematico}

In questo paragrafo discuteremo il modello del \emph{pendolo matematico}.
Osserveremo in particolare come partendo da un sistema bidimensionale non conservativo e restringendoci all'unica direzione non banale, otterremo un'equazione conservativa unidimensionale.

Il modello è costituito da una massa puntiforme \(m\), collegata attraverso un filo inestensibile di lunghezza \(l\) al punto di sospensione.
Il moto di \(m\) sarà pertanto un moto circolare di raggio \(l\). L'equazione del moto sarà pertanto
\[
	m\,\ddot{\vec{x}} = m\,\vec{g} + \vec{T} \qquad\qquad %!TEX root = ../../main.tex
\tikz[baseline=-0.5cm, scale=1.5]{
	\coordinate (O) at (0,0);
	\coordinate (M) at (-60:2);
	\draw [-latex, thick, gray] (O) -- (1,0);
	\draw [-latex, thick, gray] (O) -- (0,1);

	\draw[dashed, help lines] (O) -- (0, -2);
	\draw (O) -- (-60:2);
	\draw[dashed, help lines] (235:2) arc[radius=2, start angle=235, end angle= 305];
	\draw (270:0.3) arc (270:300:0.3);
	\draw[->, thick] (M) -- (-60:1.5);
	\draw[->, thick] (M) -- ++ (0, -0.5);

	\fill (M) circle (0.03);

	\node[above, font=\footnotesize, gray] at (1,0) {\(x\)};
	\node[right, font=\footnotesize, gray] at (0,1) {\(y\)};
	\node at (-75:0.45) {\(\q\)};
	\node[left] at (-60:1) {\(l\)};
	\node[left] at (-60:2) {\(m\)};
	\node[right] at ($(-60:2) + (0,-0.5)$) {\(m\,\vec{g}\)};
	\node[above right] at (-60:1.5) {\(\vec{T}\)};
}
\]
dove \(\vec{g}\) è l'accelerazione di gravità rivolta verso il basso e \(\vec{T}\) è la tensione del filo. In particolare, se definiamo con \(\q\) l'angolo che viene individuato dalla massa con la normale al punto di sospensione, avremo
\[
	\vec{x} = l\,\begin{pmatrix}\sin \q\\-\cos\q\end{pmatrix}, \qquad \vec{g} = \begin{pmatrix}0\\-g\end{pmatrix}, \qquad \vec{T} = T\,\begin{pmatrix}-\sin\q\\\cos \q\end{pmatrix}.
\]
Definiamo le componenti radiali \(\hat{e}_r\) e tangenziali \(\hat{e}_t\) del punto materiale:
\[
	\hat{e}_r = \begin{pmatrix}\sin\q\\-\cos\q\end{pmatrix} \qquad\text{e}\qquad \hat{e}_t = \begin{pmatrix}\cos\q\\\sin\q\end{pmatrix} \qquad \qquad %!TEX root = ../../main.tex
\tikz[baseline=-2.5cm, scale=2]{
	\coordinate (O) at (0,0);

	\draw[dashed, help lines] (O) -- (0, -1.5);
	\draw (O) -- (-60:1.5);
	\draw[dashed, help lines] (230:1.5) arc[radius=1.5, start angle=230, end angle= 300];
	\draw (270:0.3) arc (270:300:0.3);
	\draw[dashed] (-60:1.5) -- ++ (0,-1);
	\draw[dashed] (-60:1.5) -- ++ (1, 0);
	\draw[->, thick] (-60:1.5) -- (-60:2.5);
	\draw[->, thick] (-60:1.5) -- ++ (30:1);
	\draw ([shift={(-60:1.5)}]270:0.3) arc[radius=0.3, start angle=270, end angle= 300];
	\draw ([shift={(-60:1.5)}]0:0.3) arc[radius=0.3, start angle=0, end angle= 30];

	\node at (-75:0.5) {\(\q\)};
	\node at ($(-60:1.5) + (15:0.5)$) {\(\q\)};
	\node at ($(-60:1.5) + (-75:0.5)$) {\(\q\)};
	\node[right] at (-60:2.5) {\(\hat{e}_r\)};
	\node[above] at ($(-60:1.5) + (30:1)$) {\(\hat{e}_t\)};
}
\]
Proiettiamo il moto lungo queste componenti, le quali costituiscono chiaramente una base ortonormale. Per il vettore posizione avremo
\[
	\vec{x} = l\,	\begin{pmatrix}
		\sin\q \\
		-\cos \q	
	\end{pmatrix}
	= l\,\hat{e}_r \implies
	\dot{\vec{x}} = l\,\dot{\q}\, 	\begin{pmatrix}
		\cos\q \\
		\sin\q
	\end{pmatrix}
	= l\,\dot{\q}\,\hat{e}_t,										
\]
da cui
\[
	\ddot{\vec{x}} = l\,\ddot{\q}\,\hat{e}_t + l\,\dot{\q}^2 	\begin{pmatrix}
		-\sin\q \\
		\cos\q
	\end{pmatrix}
	= l\,\ddot{\q}\,\hat{e}_t - l\,\dot{\q}^2 \hat{e}_r.												
\]
La tensione si proietta direttamente, infatti
\[
	\vec{T} = T\,\begin{pmatrix}-\sin\q\\\cos \q\end{pmatrix} = -T\,\hat{e}_r.
\]
Resta infine da scrivere \(\vec{g}\) nella nuova base, per farlo proiettiamo direttamente tramite pordotto scalare:
\[
	\vec{g}\sdot \hat{e}_r = \begin{pmatrix}0\\-g\end{pmatrix}\sdot\begin{pmatrix}\sin\q\\-\cos\q\end{pmatrix} = g\,\cos\q \qquad\text{e}\qquad
	\vec{g}\sdot \hat{e}_t = \begin{pmatrix}0\\-g\end{pmatrix}\sdot\begin{pmatrix}\cos\q\\\sin\q\end{pmatrix} = -g\,\sin\q;
\]
da cui
\[
	\vec{g} = g\,\cos\q\,\hat{e}_r - g\,\sin\q\,\hat{e}_t.
\]
Sostituendo nell'equazione del moto otteniamo
\[
	m\,\ddot{\vec{x}} = m\,\vec{g}+\vec{T} \iff m\,l\,\ddot{\q}\,\hat{e}_t - m\,l\,\dot{\q}^2 \hat{e}_r = m\,g\,\cos\q\,\hat{e}_r - m\,g\,\sin\q\,\hat{e}_t - T\,\hat{e}_r,
\]
che può essere espresso tramite il sistema
\[
	\left\{\begin{aligned}
		m\,l\,\ddot{\q}   & = -m\,g\,\sin\q     \\
		-m\,l\,\dot{\q}^2 & = m\,g\,\cos\q - T
	\end{aligned}\right.
\]
La seconda ci dice immediatamente che la tensione dipende dalla velocità angolare, pertanto non è posizionale e tantomeno conservativa.

\begin{remark}{Equazione del pendolo matematico}{equazionePendoloMatematico}
	\[
		m\,l\,\ddot{\q} = -m\,g\,\sin\q.
	\]
\end{remark}

\begin{notz}
	Alternativamente possiamo scrivere l'equazione del pendolo matematico come
	\[
		\ddot{\q} = -\frac{g}{l}\,\sin\q = -\w^2\sin\q.
	\]
	Osserviamo che ha senso identificare \(g/l\) come una frequenza al quadrato in quanto ha le dimensioni del reciproco di un tempo al quadrato, infatti
	\[
		\left[ \frac{g}{l} \right]  = \left[ \frac{l}{t^2} \right]\,\frac{1}{[l]} = \frac{1}{[t]^2}.
	\]
\end{notz}
\noindent
Cerchiamo di esprimere quest'ultima equazione attraverso un "potenziale"\graffito{non possiamo parlare propriamente di potenziale in quanto le dimensioni fisiche sono diverse}, come avviene per le equazioni del moto conservative:
\[
	\ddot{\q} = -\w^2\sin\q = -\w^2 \frac{\pd}{\pd\q}(1-\cos\q) = -u'(\q),
\]
dove \(u(\q) = \w^2(1-\cos\q)\). Possiamo quindi scrivere l'energia del sistema, che in quanto conservativa sarà costante:
\[
	\frac{\dot{\q}^2}{2} + u(\q) = cost.
\]
Possiamo ricondurci alle dimensioni di un'energia moltiplicando ambo i membri per \(m\,l^2\), ottenendo
\[
	m\,l^2 \frac{\dot{\q}^2}{2} + m\,l^2 u(\q) = E,
\]
da cui

\begin{remark}{Equazione dell'energia del sistema}{equazioneEnergiaSistema}
	\[
		\frac{\dot{\q}^2}{2}+u(\q) = \frac{E}{m\,l^2}.
	\]
\end{remark}
\subsection{Piano delle fasi}

L'equazione del pendolo ci dice che il moto di \(\q(t)\) a energia \(E\), si svolge sulla curva di livello di equazione
\[
	\dot{\q} = \pm\sqrt{2 \left( \frac{E}{m\,l^2}-u(\q) \right)}. \tag{\(\star\)}
\]
Per visualizzare meglio la struttura delle soluzioni, possiamo considerare il piano \((\q,\dot{\q})\) detto piano delle fasi. Nel nostro caso esso corrisponde all'insieme delle curve di livello \(\Gamma_E\)
\[
	\Gamma_E = \Set{(\q,\dot{\q}) | \frac{\dot{\q}^2}{2}+u(\q) = \frac{E}{m\,l^2}},
\]
al variare della costante \(E\).

\begin{oss}
	Il procedimento che applicheremo a questo scopo, è in generale applicabile per ottenere il grafico qualitativo di altri piani di fasi.
\end{oss}
\noindent
Dal momento che \(\q\) è una variabile angolare, possiamo restringerci ad un dominio di ampiezza \(2\p\) come \([-\p,\p]\).

\begin{oss}
	Analogamente potremmo dire che \(u(\q)\) è periodica con \(T=2\p\).
\end{oss}
\noindent
Per prima cosa facciamo un grafico qualitativo di \(u(\q) = \w^2(1-\cos\q)\):
\[
	\tikz{
		\draw[-latex] (-3.5, 0) -- (3.5,0) node[above] {\(\q\)};
		\draw[-latex] (0,-1) -- (0,2.5) node[right] {\(u(\q)\)};
		\draw[domain=-pi-0.2:pi+0.2,smooth,variable=\x, compl, thick] plot ({\x},{1-cos(deg(\x))});
		\draw[dashed, thin] (-pi, 0) -- (-pi, 2);
		\draw[dashed, thin] (pi, 0) -- (pi, 2);
		\draw[dashed, thin] (-pi, 2) -- (pi, 2);
		\draw[dashed, thin] (-pi/3, 1/2) -- (pi/3, 1/2);
		\draw[dashed, thin] (-pi/3, 0) -- (-pi/3, 1/2);
		\draw[dashed, thin] (pi/3, 0) -- (pi/3, 1/2);
		\node[below, font=\small] at (-pi, 0) {\(-\p\)};
		\node[below, font=\small] at (pi, 0) {\(\p\)};
		\node[below left, font=\small] at (0, 2) {\(2\w^2\)};
		\node[below, font=\small] at (-pi/3,0) {\(\q_-\)};
		\node[below, font=\small] at (pi/3, 0) {\(\q_+\)};
		\node[above right, font=\small] at (0,1/2) {\(\frac{E}{m\,l^2}\)};
	}
\]
Come si evince anche dal grafico, \(u(\q)\) ha un punto di equilibrio stabile in \(0\) e due punti di equilibrio instabili in \(\pm\p\).
%%%%%%%%%%%%%%%%%%%%%%%%%%%%%%%%%%%%%%%%%
%
%LEZIONE 15/03/2017 - TERZA SETTIMANA (2)
%
%%%%%%%%%%%%%%%%%%%%%%%%%%%%%%%%%%%%%%%%%
Dobbiamo valutare tre casi al variare di \(\frac{E}{m\,l^2}\), che chiameremo rispettivamente sottocritico, critico e sovracritico.

Per \(0<\frac{E}{m\,l^2}<2\w^2\), che è il caso disegnato nel grafico precedente, avremo che la porzione di grafico rilevante per la curva di livello, sarà compresa fra \(\q_-\) e \(\q_+\). In particolare avremo
\[
	\q_+ = \arccos \left( 1- \frac{E}{m\,l^2 w^2} \right) \qquad\text{e}\qquad \q_- = -\q_+.
\]
In particolare il grafico del radicando di \((\star)\) e la conseguente curva di livello saranno,
\[
	\tikz[baseline=-0.5ex]{
		\draw[-latex] (-2, 0) -- (2,0) node[above] {\(\q\)};
		\draw[-latex] (0,-1) -- (0,1.5) node[right] {\(2\left(\frac{E}{m\,l^2}-u(\q)\right)\)};
		\draw[domain=-pi/3:pi/3,smooth,variable=\x, compl, thick] plot ({\x},{cos(deg(\x))-1/2});
		\node[below, font=\small] at (-pi/3,0) {\(\q_-\)};
		\node[below, font=\small] at (pi/3,0) {\(\q_+\)};
	} \qquad\text{e}\qquad
	\tikz[baseline=-0.5ex]{
		\draw[-latex] (-2, 0) -- (2,0) node[above] {\(\q\)};
		\draw[-latex] (0,-1) -- (0,1.5) node[right] {\(\dot{\q}\)};
		\draw[domain=-pi/3+0.0001:pi/3-0.0001,smooth,variable=\x, compl, thick] plot ({\x},{sqrt(cos(deg(\x))-1/2)});
		\draw[domain=-pi/3+0.0001:pi/3-0.0001,smooth,variable=\x, compl, thick] plot ({\x},{-sqrt(cos(deg(\x))-1/2)});
		\draw[dashed, thin] (-pi/2, -1) -- (-pi/2, 1);
		\draw[dashed, thin] (pi/2, -1) -- (pi/2, 1);
		\node[below right, font=\small] at (-pi/3,0) {\(\q_-\)};
		\node[below left, font=\small] at (pi/3,0) {\(\q_+\)};
		\node[below left, font=\small] at (-pi/2,0) {\(-\p\)};
		\node[below right, font=\small] at (pi/2,0) {\(\p\)};
	}
\]
Come possiamo vedere il raccordo nella curva di livello è liscio, questo poiché, non essendo punti critici di \(u\), la derivata del radicando in \(\q_-\) e \(\q_+\) è diversa da zero.

\begin{oss}
	In realtà il raccordo non è solo continuo ma addirittura \(C^{\infty}\). Questo poiché l'espressione si ottiene come soluzione di un problema di Cauchy i cui dati sono \(C^{\infty}\) e la cui soluzione è pertanto parametrizzata come punti \(C^{\infty}\).
\end{oss}
\noindent
Questa sorta di ellisse si espande all'aumentare di \(E\) fino al raggiungimento del caso critico, in cui \(\frac{E}{m\,l^2}= 2\w^2\). In questo caso \((\star)\) assume la forma
\[
	\dot{\q} = \pm \sqrt{2(2\w^2-\w^2+\w^2\cos\q)} = \pm\sqrt{2\w^2(1+\cos\q)}.
\]
In questo caso i grafici diventano
\[
	\tikz[baseline=-0.5ex]{
		\draw[-latex] (-2, 0) -- (2,0) node[above] {\(\q\)};
		\draw[-latex] (0,-1) -- (0,1.5) node[right] {\(2\w^2(1+\cos\q)\)};
		\draw[scale=1/2, domain=-pi:pi,smooth,variable=\x, compl, thick] plot ({\x},{1+cos(deg(\x))});
		\node[below, font=\small] at (-pi/2,0) {\(-\p\)};
		\node[below, font=\small] at (pi/2,0) {\(\p\)};
	} \qquad\text{e}\qquad
	\tikz[baseline=-0.5ex]{
		\draw[-latex] (-2, 0) -- (2,0) node[above] {\(\q\)};
		\draw[-latex] (0,-1) -- (0,1.5) node[right] {\(\dot{\q}\)};
		\draw[scale=1/2, domain=-pi+0.0001:pi-0.0001,smooth,variable=\x, compl, thick] plot ({\x},{sqrt(1+cos(deg(\x)))});
		\draw[scale=1/2, domain=-pi+0.0001:pi-0.0001,smooth,variable=\x, compl, thick] plot ({\x},{-sqrt(1+cos(deg(\x)))});
		\draw[dashed, thin] (-pi/2, -1) -- (-pi/2, 1);
		\draw[dashed, thin] (pi/2, -1) -- (pi/2, 1);
		\node[below left, font=\small] at (-pi/2,0) {\(-\p\)};
		\node[below right, font=\small] at (pi/2,0) {\(\p\)};
	}
\]
Osserviamo che in questo caso non vi è raccordo negli zeri. Questo poiché il termine dominante nell'espansione di taylor di \(2\w^2(1+\cos\q)\) negli zeri \(\pm\p\), è
\[
	\w^2(\q\mp\p)^2,
\]
pertanto, passando alla radice nella curva di livello, il termine dominante è lineare e il raccordo non sarà pertanto lisce.

\begin{oss}
	Questo mancato raccordo non nega l'osservazione precedente sul raccordo \(C^{\infty}\). Questa situazione vi si configura osservando che, in questo caso, il moto si compie unicamente nel semipiano superiore o in quello inferiore, senza mai attraversare l'asse \(\q\).
\end{oss}
\noindent
Discutiamo infine il caso sovracritico in cui \(\frac{E}{m\,l^2}>2\w^2\). In tal caso i grafici qualitativi saranno
\[
	\tikz[baseline=-0.5ex]{
		\draw[-latex] (-2, 0) -- (2,0) node[above] {\(\q\)};
		\draw[-latex] (0,-1) -- (0,1.5) node[right] {\(2\left(\frac{E}{m\,l^2}-u(\q)\right)\)};
		\draw[scale=1/2, domain=-pi:pi,smooth,variable=\x, compl, thick] plot ({\x},{1.5+cos(deg(\x))});
		\draw[dashed, thin] (-pi/2, -1) -- (-pi/2, 1);
		\draw[dashed, thin] (pi/2, -1) -- (pi/2, 1);
		\draw[dashed, thin] (-pi, 0.25) -- (pi, 0.25);
		\node[below left, font=\small] at (-pi/2,0) {\(-\p\)};
		\node[below right, font=\small] at (pi/2,0) {\(\p\)};
	} \qquad\text{e}\qquad
	\tikz[baseline=-0.5ex]{
		\draw[-latex] (-2, 0) -- (2,0) node[above] {\(\q\)};
		\draw[-latex] (0,-1) -- (0,1.5) node[right] {\(\dot{\q}\)};
		\draw[scale=1/2, domain=-pi+0.0001:pi-0.0001,smooth,variable=\x, compl, thick] plot ({\x},{sqrt(1.5+cos(deg(\x)))});
		\draw[scale=1/2, domain=-pi+0.0001:pi-0.0001,smooth,variable=\x, compl, thick] plot ({\x},{-sqrt(1.5+cos(deg(\x)))});
		\draw[dashed, thin] (-pi/2, -1) -- (-pi/2, 1);
		\draw[dashed, thin] (pi/2, -1) -- (pi/2, 1);
		\node[below left, font=\small] at (-pi/2,0) {\(-\p\)};
		\node[below right, font=\small] at (pi/2,0) {\(\p\)};
	}
\]
Possiamo infine riepilogare l'andamento qualitativo del moto sul piano delle fasi
\[
	%!TEX root = ../../main.tex
\tikz{
	\draw[-latex] (-4.5, 0) -- (4.5,0) node[above] {\(\q\)};
	\draw[-latex] (0,-3.5) -- (0,3.5) node[right] {\(\dot{\q}\)};

	%Sottocritici
	\draw[domain=-0.643:0.643,variable=\x, compl, thick,
	postaction={decorate, decoration={markings, mark=between positions 0.25 and 0.75 step 0.5 with {\arrow{>}}}}
	] plot ({\x},{sqrt(-1.6+2*cos(deg(\x)))});
	\draw[domain=-0.643:0.643,variable=\x, compl, thick,
	postaction={decorate, decoration={markings, mark=between positions 0.25 and 0.75 step 0.5 with {\arrow{<}}}}
	] plot ({\x},{-sqrt(-1.6+2*cos(deg(\x)))});
	\draw[domain=-pi/2+0.0001:pi/2-0.0001,variable=\x, compl, thick,
	postaction={decorate, decoration={markings, mark=between positions 0.25 and 0.75 step 0.5 with {\arrow{>}}}}
	] plot ({\x},{sqrt(2*cos(deg(\x)))});
	\draw[domain=-pi/2+0.0001:pi/2-0.0001,variable=\x, compl, thick,
	postaction={decorate, decoration={markings, mark=between positions 0.25 and 0.75 step 0.5 with {\arrow{<}}}}
	] plot ({\x},{-sqrt(2*cos(deg(\x)))});

	%Critico
	\draw[domain=-pi+0.0001:pi-0.0001,variable=\x, compl, thick,
	postaction={decorate, decoration={markings, mark=between positions 0.25 and 0.75 step 0.5 with {\arrow{>}}}}
	] plot ({\x},{sqrt(2+2*cos(deg(\x)))});
	\draw[domain=-pi+0.0001:pi-0.0001,variable=\x, compl, thick,
	postaction={decorate, decoration={markings, mark=between positions 0.25 and 0.75 step 0.5 with {\arrow{<}}}}
	] plot ({\x},{-sqrt(2+2*cos(deg(\x)))});

	%Sovracritico
	\draw[domain=-pi:pi,variable=\x, compl, thick,
	postaction={decorate, decoration={markings, mark=between positions 0.25 and 0.75 step 0.5 with {\arrow{>}}}}
	] plot ({\x},{sqrt(3.4+2*cos(deg(\x)))});
	\draw[domain=-pi:pi,variable=\x, compl, thick,
	postaction={decorate, decoration={markings, mark=between positions 0.25 and 0.75 step 0.5 with {\arrow{<}}}}
	] plot ({\x},{-sqrt(3.4+2*cos(deg(\x)))});
	\draw[domain=-pi:pi,variable=\x, compl, thick,
	postaction={decorate, decoration={markings, mark=between positions 0.25 and 0.75 step 0.5 with {\arrow{>}}}}
	] plot ({\x},{sqrt(5+2*cos(deg(\x)))});
	\draw[domain=-pi:pi,variable=\x, compl, thick,
	postaction={decorate, decoration={markings, mark=between positions 0.25 and 0.75 step 0.5 with {\arrow{<}}}}
	] plot ({\x},{-sqrt(5+2*cos(deg(\x)))});

	%Abbellimenti
	\draw[dashed, thin] (-pi,-3.5) -- (-pi, 3.5);
	\draw[dashed, thin] (pi,-3.5) -- (pi, 3.5);

	\node[below left, font=\small] at (-pi, 0) {\(-\p\)};
	\node[below right, font=\small] at (pi, 0) {\(\p\)};
}
\]
dove il verso delle curve è dato dal segno di \(\dot{\q}\).
Esso è infatti positivo quando la velocità angolare si trova nel semipiano superiore e, viceversa, negativo quando è in quello inferiore.
\subsection{Soluzioni analitiche}
Discutiamo adesso le soluzioni analitiche, anche dette \emph{soluzioni per quadrature}.
Cominciamo con il caso sottocritico, che ricordiamo essere quello in cui
\[
	0 < \frac{E}{m\,l^2} < 2\w^2.
\]
Scegliamo un istante iniziale \(t_0\) in cui \(\dot{\q}(t_0)>0\).
Per \(t_0<t<t_1\), dove \(t_1\) è un istante ancora da determinare in cui \(\q(t_1)=\q_+\), avremo
\[
	\dot{\q} = +\sqrt{2 \left( \frac{E}{m\,l^2}-u(\q) \right)}.
\]
Per separazione di variabili e integrando fra \(t_0\) e \(t\) otteniamo
\[
	F\big(\q(t)\big) = \int_{\q_-}^{\q(t)} \frac{\dd \q}{\sqrt{2 \left( \frac{E}{m\,l^2}-u(\q) \right)}} = t-t_0
\]
Quest'ultima è la soluzione per quadratura valida per ogni \(t\in[t_0,t_1]\).
Da ciò possiamo anche ottenere un'espressione per \(t_1\):
\[
	\q(t_1) = \q_+ \implies t_1 = t_0 + \int_{\q_-}^{\q_+} \frac{\dd\q}{\sqrt{2 \left( \frac{E}{m\,l^2}-u(\q) \right)}}.
\]
Osserviamo inoltre che \(F(\q)\) è invertibile, in particolare da \(F\big(\q(t)\big)=t-t_0\) otteniamo

\begin{remark}{Soluzione analitica nel caso sottocritico}{soluzioneAnaliticaCasoSottocritico}
	\[
		\q(t) = F^{-1}(t-t_0).
	\]
\end{remark}

\begin{oss}
	\(F(\q_-)=0\) e \(F(\q)\) è monotona strettamente crescente per \(\q_-\le \q \le \q_+\), pertanto è invertibile.
\end{oss}
\noindent
Osservando il piano delle fasi, ci aspettiamo che \(t_1\) sia finito per il caso sottocritico, ovvero che
\[
	\int_{\q_-}^{\q_+} \frac{\dd\q}{\sqrt{2 \left( \frac{E}{m\,l^2}-u(\q) \right)}} < +\infty.
\]
Dobbiamo pertanto studiare l'andamento del denominatore. Se \(\q\) è sufficientemente vicino a \(\q_-\), avremo
\[
	\frac{E}{m\,l^2} - u(\q) = c\,(\q-\q_-) + \bO\big((\q-\q_-)^2\big) \simeq c\,(\q-\q_-).
\]
Da cui
\[
	\int_{\q_-}^{\q_-+\e} \frac{\dd\q}{\sqrt{2 \left( \frac{E}{m\,l^2}-u(\q) \right)}} \simeq \int_{\q_-}^{\q_-+\e} \frac{\dd\q}{\sqrt{c\,(\q-\q_-)}} < +\infty.
\]
E analogamente si mostra
\[
	\int_{\q_+-\e}^{\q_+} \frac{\dd\q}{\sqrt{2 \left( \frac{E}{m\,l^2}-u(\q) \right)}} < +\infty.
\]
Ripetiamo il procedimento per istanti \(t\) in cui \(\dot{\q}<0\), ovvero per \(t_1\le t \le t_2\), dove \(t_2\) è un'istante ancora da determinare in cui \(\q(t_2)=\q_-\). In questo caso avremo
\[
	\dot{\q} = -\sqrt{2 \left( \frac{E}{m\,l^2}-u(\q) \right)}.
\]
Nuovamente per separazione di variabili e integrando fra \(t_1\) e \(t\) otteniamo
\[
	G\big(\q(t)\big) = \int_{\q(t)}^{\q_+} \frac{\dd\q}{\sqrt{2 \left( \frac{E}{m\,l^2}-u(\q) \right)}} = t-t_1.
\]
Osserviamo che
\[
	G\big(\q(t)\big) = \int_{\q_-}^{\q_+} \frac{\dd\q}{\sqrt{2 \left( \frac{E}{m\,l^2}-u(\q) \right)}} - F\big(\q(t)\big),
\]
inoltre anche \(G\) è invertibile, per cui
\[
	G\big(\q(t)\big) = t-t_1 \iff \q(t) = G^{-1}(t-t_1) = F^{-1} \bigg( \int_{\q(t)}^{\q_+} \frac{\dd\q}{\sqrt{2 \left( \frac{E}{m\,l^2}-u(\q) \right)}} + t_1-t \bigg).
\]
Calcolando \(t_2\), possiamo infine determinare il periodo del moto:
\[
	t_2 = t_1 + \int_{\q_-}^{\q_+} \frac{\dd\q}{\sqrt{2 \left( \frac{E}{m\,l^2}-u(\q) \right)}} \implies T = 2\int_{\q_-}^{\q_+} \frac{\dd\q}{\sqrt{2 \left( \frac{E}{m\,l^2}-u(\q) \right)}}.
\]

L'intero procedimento può essere ripetuto per il caso sovracritico in cui \(\frac{E}{m\,l^2} > 2\w^2\).
Sia \(t_0\) l'istante in cui \(\q(t_0)=-\p\). Se \(\dot{\q}>0\) avremo
\[
	\dot{\q} = +\sqrt{2 \left( \frac{E}{m\,l^2}-u(\q) \right)}.
\]
Preso \(t>t_0\) e integrando per separazione di variabile fra questi due istanti otteniamo
\[
	f\big(\q(t)\big) = \int_{-\p}^{\q(t)} \frac{\dd \q}{\sqrt{2 \left( \frac{E}{m\,l^2}-u(\q) \right)}} = t-t_0.
\]
Anche in questo caso \(f\) è monotona strettamente crescente e pertanto invertibile, da cui

\begin{remark}{Soluzione analitica nel caso sovracritico}{soluzioneAnaliticaCasoSovracritico}
	\[
		\q(t) = f^{-1}(t-t_0).
	\]
\end{remark}

Osserviamo che in questo caso, come ci aspettiamo dal grafico, \(f\) è monotona crescente per ogni \(t\in\R\). Quindi l'inversa è globale.
Il caso \(\dot{\q}<0\) si descrive in modo del tutto analogo.

Discutiamo infine il caso critico in cui \(\frac{E}{m\,l^2}=2\w^2\). In questo caso saremo in grado di fornire una soluzione esplicita.
Scegliamo la determinazione positiva \(\dot{\q}>0\) e l'istante iniziale \(t_0=0\) per cui
\[
	\q(0) = 0 \qquad\text{e}\qquad \dot{\q}(0) = 2\w.
\]
La curva di livello sarà data da
\[
	\dot{\q} = +\sqrt{2\w^2(1+\cos\q)} = \sqrt{2}\w\sqrt{1+\cos\q},
\]
ricordando che \(\frac{1+\cos\q}{2} = \cos^2 \frac{\q}{2}\), otteniamo
\[
	\dot{\q} = 2\w\,\cos \frac{\q}{2},
\]
dove la radice è certamente positiva in quanto
\[
	-\p<\q<\p \implies \cos \frac{\q}{2} > 0.
\]
Procediamo integrando fra \(0\) e \(t\) per separazione di variabili:
\[
	\begin{split}
		2\w\,t & = \int_0^{\q(t)} \frac{\dd\q}{\cos \frac{\q}{2}} = 2\int_0^{\q(t)/2} \frac{\dd x}{\cos x} = 2\int_0^{\q(t)/2} \frac{\cos x\,\dd x}{\cos^2 x} = 2\int_0^{\q(t)/2} \frac{\dd(\sin x)}{1-\sin^2 x}\graffito{\(x=\q/2\)}\\
		& = 2\int_0^{\sin \frac{\q(t)}{2}} \frac{\dd y}{1-y^2} = \int_0^{\sin \frac{\q(t)}{2}} \left( \frac{1}{1+y}+\frac{1}{1-y} \right)\,\dd y = \left[ \ln \frac{1+y}{1-y} \right]_0^{\sin \frac{\q(t)}{2}}\graffito{\(y=\sin x\)}\\
		& = \ln \frac{1+\sin \frac{\q(t)}{2}}{1-\sin \frac{\q(t)}{2}}.
	\end{split}
\]
Questa soluzione, oltre ad essere esplicita, è facilmente invertibile:
\[
	\begin{split}
		\ln \frac{1+\sin \frac{\q(t)}{2}}{1-\sin \frac{\q(t)}{2}} = 2\w\,t & \iff \frac{1+\sin \frac{\q(t)}{2}}{1-\sin \frac{\q(t)}{2}} = e^{2\w\,t} \iff 1+\sin \frac{\q(t)}{2} = e^{2\w\,t} \left( 1-\sin \frac{\q(t)}{2} \right)\\
		& \iff \sin \frac{\q(t)}{2} + e^{2\w\,t} \sin \frac{\q(t)}{2} = e^{2\w\,t}-1 \iff \sin \frac{\q(t)}{2} = \frac{e^{2\w\,t}-1}{e^{2\w\,t}+1}\\
		& \iff \sin \frac{\q(t)}{2} = \frac{e^{\w\,t}-e^{-\w\,t}}{e^{\w\,t}+e^{-\w\,t}} = \tanh(\w\,t),
	\end{split}
\]
da cui

\begin{remark}{Soluzione analitica nel caso critico}{soluzioneAnaliticaCasoCritico}
	\[
		\q(t) = 2\arcsin\big(\tanh(\w\,t)\big).
	\]
\end{remark}

\begin{oss}
	Come ci aspettavamo dal piano delle fasi, nel caso critico si ha
	\[
		\q(t) \longrightarrow \pm \p \qquad\text{per }t \to \pm \infty.
	\]
\end{oss}
%%%%%%%%%%%%%%%%%%%%%%%%%%%%%%%%%%%%%%%%%%
%
%LEZIONE 20/03/2017 - QUARTA SETTIMANA (1)
%
%%%%%%%%%%%%%%%%%%%%%%%%%%%%%%%%%%%%%%%%%%
\section{Moto unidimensionale generico}

In questo paragrafo descriveremo i passaggi effettuati nel caso del pendolo matematico per un sistema unidimensionale generico.

Consideriamo il moto in una dimensione descritto dall'equazione di Newton
\[
	m\,\ddot{x} = F(x) = -U'(x).
\]
Abbiamo \hyperref[oss:conservazioneUnidim]{già osservato} come, nel caso unidimensionale, quasi ogni funzione reale ammetta una primitiva.
Pertanto, in ogni caso che ha senso di essere considerato, l'energia meccanica
\[
	E = m\,\frac{\dot{x}}{2} + U(x),
\]
è una grandezza conservata.
Esplicitando \(\dot{x}\) è possibile ricavare, seppur non sempre in forma esplicita, l'equazione del moto
\[
	\dot{x} = \pm \sqrt{\frac{2}{m}\big(E-U(x)\big)}
\]

\begin{defn}{Curva di livello a energia costante}{curvaLivelloEnergiaCostante}\index{Curva di livello}
	Consideriamo l'equazione del moto. Fissata l'energia \(E\), definiamo la curva di livello a energia \(E\) come
	\[
		\Sigma_E = \Set{(x,\dot{x}) | m\,\frac{\dot{x}}{2} + U(x) = E},
	\]
	ovvero i punti che soddisfano l'equazione del moto.
\end{defn}
\noindent
L'analisi qualitativa del moto di una particella sottoposta al potenziale \(U\), si ottiene graficando il piano delle fasi \((x,\dot{x})\). Di seguito descriviamo il procedimento per punti:
\begin{enumerate}
	\item Graficare \(U(x)\).
	\item Identificare i punti critici del potenziale \(U\).
	      Ad essi corrispondono i cosiddetti \emph{livelli di energia critici}.
	\item Disegnare i moti delle separatrici, cioè i moti in corrispondenza dei livelli di energia critici per massimi o flessi.
	      Il grafico si ottiene da quello di \(U\), è sufficiente considerare la porzione di piano al di sotto della retta orizzontale alla quota del livello di energia critico, ma riflesso rispetto a quest'ultima.
	      Nei punti in cui il grafico interseca la retta bisogna considerare diverse possibilità:
	      \begin{itemize}
		      \item Se il punto non è instabile, la tangente alla curva sarà verticale
		      \item Se il punto è uno zero quadratico (tipicamente un massimo), la curva avrà una singolarità di punto angoloso.
		      \item Se il punto è un flesso, cioè uno zero almeno cubico, la curva avrà una tangente orizzontale.
	      \end{itemize}
	\item Orientare i rami secondo il segno di \(\dot{x}\). Di conseguenza quelli del semipiano superiore si sposteranno verso destra, mentre quelli del semipiano inferiore verso sinistra.
	\item Descrivere qualitativamente i moti, e disegnarli, per ogni livello di energia.
	\item Se richiesto, ricavare la legge oraria, che in forma implicita è data da
	      \[
		      t-t_0 = \int_{x_0}^{x(t)} \frac{\dd x}{\sqrt{\frac{2}{m}\big(E-U(x)\big)}},
	      \]
	      dove \(x_0\) è un punto opportuno per il moto (tipicamente il punto di inversione della velocità o l'infinito).
	      Se non fosse possibile calcolare la legge oraria in forma esplicita, è sempre possibile fare le seguenti analisi qualitative:
	      \begin{description}
		      \item[Curve chiuse lisce] Si tratta di oscillazioni periodiche limitate. Il cui periodo è dato da
			      \[
				      T = 2\int_{x_1}^{x_2} \frac{\dd x}{\sqrt{\frac{2}{m}\big(E-U(x)\big)}},
			      \]
			      dove \(x_1,x_2\) sono i punti di inversione della velocità.
			      L'integrale è singolare per \(U(x)=E\), ma converge.
		      \item[Curve aperte lisce] Si tratta di moti non periodici e non limitati.
			      Le particelle vengono dall'infinito e vanno verso il punto di inversione \(x_0\).
			      In questo tipo di moti siamo interessati a capire se le particelle arrivano in tempo finito all'infinito (o ad \(x_0\)) o meno. Per stabilirlo, è sufficiente studiare la convergenza di 
			      \[
				      t_{\infty} - t_0 = \int_{x_0}^{+\infty} \frac{\dd x}{\sqrt{\frac{2}{m}\big(E-U(x)\big)}}.
			      \]
			      Se l'integrale converge il tempo per raggiungere l'infinito è finito e la soluzione risulterà locale, altrimenti il tempo sarà infinito e la soluzione globale.
			      In generale per verificare la convergenza, è sufficiente controllare se
			      \[
				      -U(x) \ge x^{2+\e}.
			      \]
		      \item[Sella separatrice] Potrebbe sembrare un moto chiuso, ma in questo caso il tempo per raggiungere la sella è sempre infinito.
			      Da questo deriva che i punti di sella sono instabili.
		      \item[Moduli separatori] La curva ha una singolarità di punto angoloso (tipicamente nei punti di massimo).
			      La particella arriva nel punto critico in un tempo infinito.
		      \item[Punti critici] Se la particella parte da un punto critico con velocità nulla, non si sposterà da quel punto.
	      \end{description}
\end{enumerate}