%!TEX root = ../../main.tex
%%%%%%%%%%%%%%%%%%%%%%%%%%%%%%%%%%%%%%%%%%
%
%LEZIONE 20/03/2017 - QUARTA SETTIMANA (1)
%
%%%%%%%%%%%%%%%%%%%%%%%%%%%%%%%%%%%%%%%%%%
\chapter{Moti in campo centrale}
\section{Introduzione}
\begin{defn}{Forza centrale}{forzaCentrale}\index{Forza centrale}
	Una forza \(\vec{F}\big(\vec{r}\big)\colon \R^n \to \R^n\) si dice \emph{centrale} se è esprimibile attraverso una forza scalare che dipende solo dalla distanza dall'origine, ovvero
	\[
		\vec{F}\big(\vec{r}\big) = F\big(\abs{\vec{r}}\big)\,\hat{r} \qquad\text{dove }\hat{r} = \frac{\vec{r}}{\abs{\vec{r}}}.
	\]
\end{defn}

\begin{defn}{Coppia di forze centrali}{coppiaForzeCentrali}
	Consideriamo il sistema composto da due punti nello spazio sottoposti a forze reciproche \(\vec{f}_{12}\) e \(\vec{f}_{21}\).
	Le due forze sono una \emph{coppia di forze centrali} se sono opposte, se dipendono solo dalla distanza reciproca e se sono orientate come la congiungente delle particelle. Ovvero
	\[
		\vec{f}_{12}\big(\vec{r}_1,\vec{r}_2\big) = -\vec{f}_{21}\big(\vec{r}_1,\vec{r}_2\big) = F\big(\abs{\vec{r}_1-\vec{r}_2}\big)\,\frac{\vec{r}_1-\vec{r}_2}{\abs{\vec{r}_1-\vec{r}_2}}.
	\]
\end{defn}

\begin{defn}{Forza attrattiva}{forzaAttrattiva}\index{Forza centrale!attrattiva}
	Una forza centrale \(F\big(\vec{x}\big)\) si dice \emph{attrattiva in \(\vec{x}\)} se
	\[
		F\big(\vec{x}\big) < 0.
	\]
\end{defn}

\begin{notz}
	Se vale
	\[
		F\big(\vec{x}\big) < 0 \,\fa \vec{x},
	\]
	diremo che \(F\) è puramente attrattiva.
\end{notz}

\begin{defn}{Forza repulsiva}{forzaRepulsiva}\index{Forza centrale!repulsiva}
	Una forza centrale \(F\big(\vec{x}\big)\) si dice \emph{repulsiva in \(\vec{x}\)} se
	\[
		F\big(\vec{x}\big) > 0.
	\]
\end{defn}

\begin{notz}
	Se vale
	\[
		F\big(\vec{x}\big) > 0 \,\fa \vec{x},
	\]
	diremo che \(F\) è puramente repulsiva.
\end{notz}

\begin{defn}{Centro di massa}{centroMassa}\index{Centro di massa}
	In un sistema di \(n\) particelle, si definisce \emph{centro di massa} la posizione
	\[
		\vec{r}_{CM} = \frac{\sum m_i \vec{r}_i}{\sum m_i}.
	\]
\end{defn}

\begin{defn}{Massa ridotta}{massaRidotta}\index{Massa ridotta}
	In un sistema di \(n\) particelle, chiamiamo \emph{massa ridotta} la quantità
	\[
		m = \frac{\sum m_i}{\prod m_i}.
	\]
\end{defn}

\begin{oss}
	Tale quantità soddisfa la relazione
	\[
		\frac{1}{m} = \sum_{i=1}^n \frac{1}{m_i}
	\]
\end{oss}

Nello studio dei moti in campo centrale si presentano due possibili scenari:
\begin{itemize}
	\item Una singola particella soggetta ad una forza centrale, la cui equazione sarà
	      \[
		      m\,\ddot{\vec{r}} = \vec{F}\big(\vec{r}\big) = F\big(\abs{\vec{r}}\big)\,\frac{\vec{r}}{\abs{\vec{r}}},
	      \]
	      con \(m>0\) e \(\vec{r}\in\R^3\).
	\item Due particelle soggette ad una coppia di forze centrali, le cui equazioni saranno
	      \[
		      \begin{cases}
			      m_1\ddot{\vec{r}}_1 = \vec{f}_{12}\big(\vec{r}_1,\vec{r}_2\big) = F\big(\abs{\vec{r}_1-\vec{r}_2}\big)\,\frac{\vec{r}_1-\vec{r}_2}{\abs{\vec{r}_1-\vec{r}_2}} \\
			      m_2\ddot{\vec{r}}_2 = \vec{f}_{21}\big(\vec{r}_1,\vec{r}_2\big) = F\big(\abs{\vec{r}_1-\vec{r}_2}\big)\,\frac{\vec{r}_2-\vec{r}_1}{\abs{\vec{r}_1-\vec{r}_2}}
		      \end{cases}
	      \]
	      dove \(m_1,m_2>0\) e \(\vec{r}_1,\vec{r}_2\in \R^3\).
\end{itemize}

\section{Riconducibilità al caso di singola particella}

Valutiamo come sia possibile ricondurre il secondo caso al primo.
Abbiamo già osservato come il sistema si presenta:
\[
	\begin{cases}
		m_1\ddot{\vec{r}}_1 = F\big(\abs{\vec{r}_1-\vec{r}_2}\big)\,\frac{\vec{r}_1-\vec{r}_2}{\abs{\vec{r}_1-\vec{r}_2}} \\
		m_2\ddot{\vec{r}}_2 = F\big(\abs{\vec{r}_1-\vec{r}_2}\big)\,\frac{\vec{r}_2-\vec{r}_1}{\abs{\vec{r}_1-\vec{r}_2}}
	\end{cases} \qquad \qquad
	\tikz[baseline=-0.5ex]{
		\coordinate (A) at (1,0.5) ;
		\coordinate (B) at (-1,-0.5);
		\fill (A) circle (0.03);
		\fill (B) circle (0.03);
		\draw[dashed] (A) -- (B);
		\draw[->] (A) -- (3,1.5);
		\draw[->] (B) -- (-3,-1.5);
		\node[below=0.1cm] at (A) {\(m_2\)};
		\node[below=0.1cm] at (B) {\(m_1\)};
		\node[above left] at (3,1.5) {\(\vec{f}_{21}\)};
		\node[above=0.6cm, right] at (-3,-1.5) {\(\vec{f}_{12}\)};
	}
\]
Le due forze sono opposte, quindi sommando le due equazioni si ottiene
\[
	m_1\ddot{\vec{r}}_1 + m_2\ddot{\vec{r}}_2 = \vec{0}.
\]
Moltiplicando e dividendo per la massa totale \(m_1+m_2\) si ottiene l'espressione con il centro di massa
\[
	(m_1+m_2)\,\ddot{\vec{R}} = \vec{0}.
\]
Da ciò segue che l'accelerazione del centro di massa è nulla, pertanto la sua legge oraria si esprime semplicemente tramite
\[
	\vec{R}(t) = \vec{R}(0) + \dot{\vec{R}}(0)\,t.
\]
Tornando al sistema iniziale, dividiamo ogni equazione per la rispettiva massa e sottraiamo la prima alla seconda:
\[
	\ddot{\vec{r}}_2 - \ddot{\vec{r}}_1 = \left( \frac{1}{m_2}+\frac{1}{m_1} \right)\,F\big(\abs{\vec{r}_1-\vec{r}_2}\big)\,\frac{\vec{r}_2-\vec{r}_1}{\abs{\vec{r}_1-\vec{r}_2}}.
\]
Poniamo \(\vec{r}=\vec{r}_2-\vec{r}_1\) e sostituiamo la massa ridotta per ottenere
\[
	\ddot{\vec{r}} = \frac{1}{m}\,F\big(\abs{\vec{r}}\big)\,\hat{r}.
\]
Ci siamo pertanto ridotti al sistema
\[
	\begin{cases}
		\vec{R}(t) = \vec{R}(0) + \dot{\vec{R}}(0)\,t \\
		\ddot{\vec{r}} = \frac{1}{m}\,F\big(\abs{\vec{r}}\big)\,\hat{r}
	\end{cases}
\]
Noti \(\vec{r}(t)\) e \(\vec{R}(t)\) possiamo ricavare le leggi orarie delle due particelle attraverso il sistema
\[
	\begin{cases}
		\vec{r}(t) = \vec{r}_2(t)-\vec{r}_1(t) \\
		\vec{R}(t) = \frac{m_1}{m_1+m_2}\,\vec{r}_1(t) + \frac{m_2}{m_1+m_2}\,\vec{r}_2(t)
	\end{cases}
\]
Moltiplicando la prima riga per \(\frac{m_2}{m_1+m_2}\) e sottraendo le due righe si ottiene
\[
	\vec{R}(t) - \frac{m_2}{m_1+m_2}\,\vec{r}(t) = \frac{m_1}{m_1+m_2}\,\vec{r}_1(t) + \frac{m_2}{m_1+m_2}\,\vec{r}_1(t) = \vec{r}_1(t).
\]
Analogamente si trova \(\vec{r}_2\). Riepilogando
\[
	\begin{cases}
		\vec{r}_1 = \vec{R}- \frac{m_2}{m_1+m_2}\,\vec{r} \\
		\vec{r}_2 = \vec{R} + \frac{m_1}{m_1+m_2}\,\vec{r}
	\end{cases}
\]

\section{Conservazione dell'energia meccanica}

La forza dei moti in campo centrale è del tipo \(F\colon (0,+\infty) \to \R\).
\'E facile osservare come tali \(F\) siano sempre conservative.
Dal punto di vista fisico si considera \(\vec{r}\in \R^3\setminus{\vec{0}}\), pertanto è sufficiente verificare che \(F\) è chiusa affinché sia conservativa.
Osserviamo che
\[
	\frac{\pd}{\pd r_i} \abs{\vec{r}} = \frac{r_i}{\abs{\vec{r}}}.
\]
Da cui
\[
	\frac{\pd}{\pd r_i} \left( F\big(\abs{\vec{r}}\big)\, \frac{r_j}{\abs{\vec{r}}} \right) = F'\big(\abs{\vec{r}}\big)\,\frac{r_i}{\abs{\vec{r}}}\,\frac{r_j}{\abs{\vec{r}}} - F\big(\abs{\vec{r}}\big)\, \frac{r_j r_i}{\abs{\vec{r}}^3} = \frac{\pd}{\pd r_j}\left( F\big(\abs{\vec{r}}\big)\, \frac{r_i}{\abs{\vec{r}}} \right),
\]
pertanto \(\vec{F}\big(\vec{r}\big) = F\big(\abs{\vec{r}}\big)\,\hat{r}\) è conservativa per ogni scelta di \(F\).

\section{Potenziale radiale}

In un problema a simmetria radiale come quello che stiamo considerando, ci aspettiamo che l'energia potenziale sia unicamente funzione di \(\abs{\vec{r}}\), ovvero che

\begin{remark}{Potenziale radiale}{potenzialeRadiale}
	\[
		U\big(\vec{r}\big) = V\big(\abs{\vec{r}}\big).
	\]
\end{remark}
\noindent
Verifichiamo che scegliendo \(U\) in questo modo, troviamo una soluzione di
\[
	-\frac{\pd}{\pd\vec{r}}U\big(\vec{r}\big) = \vec{F}\big(\vec{r}\big).
\]
Ora 
\[
	-\frac{\pd}{\pd\vec{r}}V\big(\abs{\vec{r}}\big) = -V'\big(\abs{\vec{r}}\big)\,\hat{r} = F\big(\abs{\vec{r}}\big)\,\hat{r},
\]
pertanto, posto \(\r = \abs{\vec{r}}\),
\[
	V(\r) = -\int^\r F(t)\,\dd t
\]
è una qualsiasi primitiva di \(F\).

La conservatività ci permette di scrivere la legge di conservazione dell'energia meccanica
\[
	E = \frac{1}{2}\,m\,\abs{\dot{\vec{r}}}^2 + V\big(\abs{\vec{r}}\big).
\]

\begin{oss}
	Chiaramente
	\[
		\abs{\dot{\vec{r}}} \neq \frac{\dd}{\dd t} \abs{\vec{r}}.
	\]
\end{oss}

\section{Conservazione del momento angolare}

In precedenza abbiamo visto come nel caso unidimensionale, fosse sufficiente una sola relazione conservativa per risolvere il sistema.
In questo caso, in cui vi sono tre dimensioni, ci aspettiamo di doverne trovare altre affinché il sistema sia ugualmente risolubile.
In particolare, osserveremo in seguito, che ci servono tanti integrali primi quanti sono i gradi di libertà del sistema.

In questo caso saremo in grado di trovare altri 3 integrali primi, uno in più del necessario.

\begin{defn}{Momento angolare}{momentoAngolare}
	Consideriamo un punto materiale di massa \(m\) e posizione \(\vec{r}\).
	Il \emph{momento angolare} di tale punto è definito come il prodotto vettoriale del vettore posizione per il vettore quantità di moto:
	\[
		\vec{L} = \vec{r} \wedge m\,\dot{\vec{r}}.
	\]
\end{defn}
\noindent
Verifichiamo che \(\vec{L}\) è una grandezza conservata, o integrale primo, per l'equazione del moto che stiamo considerando, ovvero che \(\vec{L}=\vec{L}\big(\vec{r},\dot{\vec{r}}\big)\) è tale che
\[
	\frac{\dd}{\dd t} \vec{L}\big(\vec{r}(t),\dot{\vec{r}}(t)\big)
\]
se \(\vec{r}(t)\) è soluzione di \(m\,\ddot{\vec{r}} = F\big(\abs{\vec{r}}\big)\,\hat{r}\).
Verifichiamolo:
\[
	\frac{\dd}{\dd t}\vec{L} = \cancel{\dot{\vec{r}} \wedge m\,\dot{\vec{r}}} + \vec{r} \wedge m\,\ddot{\vec{r}} = \vec{r} \wedge m\,\ddot{\vec{r}}.
\]
Ora se \(\vec{r}(t)\) è soluzione del moto, possiamo sostituire \(m\,\ddot{\vec{r}}\), da cui
\[
	\frac{\dd}{\dd t}\vec{L} = \vec{r} \wedge F\big(\abs{\vec{r}}\big)\,\hat{r} = \vec{r} \wedge F\big(\abs{\vec{r}}\big)\,\frac{\vec{r}}{\abs{\vec{r}}} = \frac{F\big(\abs{\vec{r}}\big)}{\abs{\vec{r}}}\,\big(\vec{r}\wedge \vec{r}\big) = \vec{0}.
\]
Dalla conservazione di \(\vec{L}\) otteniamo quindi tre leggi di conservazione date dalle conservazioni delle singole componenti di \(\vec{L}\)

\subsection{Caso banale}

Supponiamo che \(\vec{L} \equiv \vec{0}\), pertanto \(\vec{r}(0) \parallel \dot{\vec{r}}(0)\).
Definiamo
\[
	\hat{n}_0 = \frac{\vec{r}(0)}{\abs{\vec{r}(0)}}
\]
la direzione iniziale di \(\vec{r}\) e \(\dot{\vec{r}}\).
Di conseguenza l'equazione del moto assume una forma molto semplice:
\[
	\vec{r}(t) = x(t)\,\hat{n}_0.
\]
Dove, sostituendo nell'equazione del moto, si ottiene che \(x(t)\) risolve
\[
	m\,\ddot{x} = F(\abs{x})
\]
in quanto \(\hat{n}_0\) è indipendente dal tempo e pertanto
\[
	\ddot{\vec{r}}(t) = \ddot{x}(t)\,\hat{n}_0.
\]
Per cui ci siamo ridotti ad un sistema conservativo unidimensionale.

\section{Riduzione dimensionale}

Supponiamo che \(\vec{L} \not\equiv \vec{0}\)

\begin{notz}
	Da qui in avanti esprimeremo \(\vec{L}\) anche in coordinate sferiche, tramite il suo modulo \(L=\abs{\vec{L}}\) e gli angoli \(\q\) e \(\j\) che lo individuano nello spazio.
\end{notz}
\noindent
Dal momento che \(\vec{L}\) si conserva, la sua direzione sarà invariante.
Questo ci dice che \(\vec{r}(t),\dot{\vec{r}}(t)\) appartengono al piano perpendicolare a \(\vec{L}\) e passante per l'origine per ogni \(t\ge 0\).

\'E quindi conveniente scegliere delle coordinate che siano legate a questo piano per descrivere il moto del sistema.
Tali coordinate \(x',y',z'\) saranno definite in modo che
\[
	\hat{z}' = \hat{L}.
\]
Nelle nuove coordinate avremo
\[
	\vec{r} = 	\begin{pmatrix}
		x'(t) \\
		y'(t) \\
		0
	\end{pmatrix}
	\qquad\text{e}\qquad
	\vec{L} = 	\begin{pmatrix}
		0 \\
		0 \\
		L
	\end{pmatrix}
\]
Ci siamo quindi ridotti ad un sistema bidimensionale.
Possiamo semplificare ulteriormente la notazione passando in coordinate polari, ottenendo
\[
	\vec{r}(t) = \r(t)\, 	\begin{pmatrix}
		\cos\q(t) \\
		\sin\q(t) \\
		0
	\end{pmatrix}
	\qquad\text{con }\r(t) = \abs{\vec{r}(t)}.
\]
Infine possiamo introdurre i versori radiale \(\hat{e}_r\) e tangenziale \(\hat{e}_t\) come mostrati nel piano \((x',y')\):
\[
	\tikz[baseline=-0.5ex, scale=2]{
		\draw[-latex] (-0.5,0) -- (2, 0) node[above] {\(x'\)};
		\draw[-latex] (0,-0.5) -- (0,2) node[right] {\(y'\)};
		\draw[->] (0,0) -- (30:1.5);
		\draw[->, thick] (30:1.5) -- ++(30:0.5) node[shift={(30:0.3)}] {\(\hat{e}_r\)};
		\draw[->, thick] (30:1.5) -- ++(120:0.5) node[shift={(120:0.3)}] {\(\hat{e}_t\)};
		\draw (0:0.3) arc[radius=0.3, start angle=0, end angle=30];
		\node[above] at (30:0.75) {\(\r\)};
		\node at (15:0.4) {\(\q\)};
	}
	\qquad\text{con }\hat{e}_r = 	\begin{pmatrix}
		\cos\q(t) \\
		\sin\q(t) \\
		0
	\end{pmatrix}
	\qquad\text{e}\qquad\hat{e}_t = 	\begin{pmatrix}
		-\sin\q(t) \\
		\cos\q(t)  \\
		0
	\end{pmatrix}
\]
Procediamo quindi a tradurre tutte le espressioni del moto nelle nuove coordinate.
Per definizione
\[
	\vec{r} = \r\,\hat{e}_r,
\]
da cui
\[
	\dot{\vec{r}} = \dot{\r}\,\hat{e}_r + \r\, \begin{pmatrix}-\sin\q\\\cos\q\\0\end{pmatrix}\,\dot{\q} = \dot{\r}\,\hat{e}_r + \r\,\dot{\q}\,\hat{e}_t.
\]
Tramite \(\dot{\vec{r}}\) possiamo calcolare \(\abs{\dot{\vec{r}}}^2\):
\[
	\abs{\dot{\vec{r}}}^2 = \dot{\r}^2 + \r^2 \dot{\q}^2.
\]
Pertanto, sostituendo nell'espressione dell'energia meccanica, otteniamo

\begin{remark}{Energia meccanica ridotta}{energiaMeccanicaRidotta}
	\[
		E = \frac{1}{2}\,m\,\dot{\r}^2 + \frac{1}{2}\,m\,\r^2\dot{\q}^2 + V(\r).
	\]
\end{remark}
\noindent
Mentre, sostituendo nell'espressione del momento angolare, otteniamo
\[
	\vec{L} = \r\,\hat{e}_r \wedge (\cancel{m\,\dot{\r}\,\hat{e}_r} + m\,\r\,\dot{\q}\,\hat{e}_t) = m\,\r^2\,\dot{\q}\,(\hat{e}_r \wedge \hat{e}_t) = m\,\r^2\,\dot{\q}\,\hat{L},
\]
dove \(\hat{e}_r \wedge \hat{e}_t=\hat{L}\) poiché sappiamo che \(\hat{e}_r, \hat{e}_t\) individuano il piano perpendicolare a \(\hat{L}\). Pertanto

\begin{remark}{Momento angolare ridotto}{momentoAngolareRidotto}
	\[
		\vec{L} = m\,\r^2\,\dot{\q}\,\hat{L}.
	\]
\end{remark}
\noindent
Traduciamo infine l'equazione del moto calcolando \(\ddot{\vec{r}}\):
\[
	\ddot{\vec{r}} = \ddot{\r}\,\hat{e}_r + \dot{\r}\,\dot{\q}\,\hat{e}_t + (\dot{\r}\,\dot{\q}+\r\,\ddot{\q})\,\hat{e}_t - \r\,\dot{\q}^2 \hat{e}_r = (\ddot{\r}-\r\,\dot{\q}^2)\,\hat{e}_r + (2\dot{\r}\,\dot{\q}+\r\,\ddot{\q})\,\hat{e}_t,
\]
sostituendo nell'equazione del moto otteniamo
\[
	m\,\ddot{\vec{r}} = m\,(\ddot{\r}-\r\,\dot{\q}^2)\,\hat{e}_r + m\,(2\dot{\r}\,\dot{\q}+\r\,\ddot{\q})\,\hat{e}_t = F(\r)\,\hat{e}_r,
\]
da cui

\begin{remark}{Equazione del moto ridotta}{equazioneMotoRidotta}
	\[
		\begin{cases}
			m\,\ddot{\r} = F(\r) + m\,\r\,\dot{\q}^2 \\
			m\,\r\,\ddot{\q} + 2m\,\dot{\r}\,\dot{\q} = 0
		\end{cases}
	\]
\end{remark}

\begin{oss}
	La seconda equazione corrisponde alla conservazione del momento angolare, infatti
	\[
		\frac{\dd}{\dd t}\vec{L} = \vec{0} \iff 0 = \frac{\dd}{\dd t}m\,\r^2\,\dot{\q} = 2m\,\r\,\dot{\r}\,\dot{\q} + m\,\r^2\ddot{\q} = \r(2m\,\dot{\r}\,\dot{\q} + m\,\r\,\ddot{\q}).
	\]
\end{oss}

\section{Potenziale efficace}

Tramite la prima equazione
\[
	m\,\ddot{\r} = F(\r) + m\,\r\,\dot{\q}^2,
\]
possiamo ricondurci ad un sistema unidimensionale che studi la componente radiale del moto.
Sostituendo l'espressione del potenziale del momento angolare nell'equazione otteniamo
\[
	\begin{split}
		m\,\ddot{\r} & = F(\r) + m\,\r\,\dot{\q}^2 = -V'(\r) + m\,\r\, \frac{L^2}{m^2\,\r^4} = -V'(\r) + \frac{L^2}{m\,\r^3}\\
		& = -\left( V(\r) + \frac{L^2}{2m\,\r^2} \right)' = -V_{eff}'(\r),
	\end{split}
\]
dove

\begin{defn}{Potenziale efficace}{potenzialeEfficace}\index{Potenziale efficace}
	In un sistema di moto, definiamo \emph{potenziale efficace} un'espressione che coinvolge il potenziale e il momento angolare del sistema.
	Nel caso di moto in campo centrale avremo
	\[
		V_{eff}(\r) = V(\r) + \frac{L^2}{2m\,\r^2}.
	\]
\end{defn}
\noindent
Da cui
\[
	m\,\ddot{\r} = F(\r) + m\,\r\,\dot{\q}^2 \iff m\,\ddot{\r} = -V_{eff}'(\r).
\]
Possiamo facilmente osservare che
\[
	E = \frac{1}{2}\,m\,\dot{\r}^2 + \frac{1}{2}\,m\,\r^2\,\dot{\q}^2 + V(\r),
\]
è l'espressione della conservazione dell'energia meccanica per il sistema \(m\,\ddot{\r} = -V_{eff}'(\r)\).
Infatti, sostituendo l'espressione di \(\dot{\q}\), che troviamo nel momento angolare
\[
	L = m\,\r^2\dot{\q} \iff \dot{\q} = \frac{L}{m\,\r^2},
\]
all'interno dell'energia meccanica, troviamo
\[
	E = \frac{1}{2}\,m\,\dot{\r}^2 + \frac{1}{2}\,m\,\r^2\,\frac{L^2}{m^2 \r^4} + V(\r) = \frac{1}{2}\,m\,\dot{\r}^2 + V_{eff}(\r).
\]
Ci siamo quindi ricondotti ad un caso unidimensionale la cui equazione conservativa è

\begin{remark}{Integrale primo del moto radiale}{integralePrimoMotoRadiale}
	\[
		E = \frac{1}{2}\,m\,\dot{\r}^2 + V_{eff}(\r).
	\]
\end{remark}

\section{Moto complessivo}

Una volta descritto il moto radiale, è possibile analizzare quello angolare per descrivere il moto complessivo.
L'espressione del moto angolare si ricava dalla relazione
\[
	L = m\,\r^2 \dot{\q} \implies \dot{\q} = \frac{L}{m\,\r^2}.
\]
In generale è utile definire la seguente quantità

\begin{remark}{Pulsazione angolare}{pulsazioneAngolare}
	\[
		w_1 = \dot{\q} = \frac{L}{m\,\r^2}.
	\]
\end{remark}

\subsection{Traiettoria banale}
Consideriamo inizialmente il caso notevole in cui il moto radiale è costante nel punto di equilibrio:
\[
	\r(t) \equiv \r_{eq}.
\]
In questo caso il moto radiale è invariante, mentre quello angolare varia uniformemente. Infatti
\[
	\dot{\q} = \frac{L}{m\,\r_{eq}^2} = \w_1
\]
che è costante.
Il moto complessivo è pertanto circolare uniforme di periodo
\[
	T_1 = \frac{2\p}{\w_1}.
\]

\subsection{Traiettoria chiusa non singolare}
Per traiettorie chiuse e non singolari, il moto è periodico 
\[
	\r(t) = \r(t+T_0) \qquad\text{con } T_0 = 2\int_{\r_-}^{\r_+} \frac{\dd\r}{\sqrt{\frac{2}{m}\big(E-V_{eff}(\r)\big)}}
\]
Osserviamo che, dal momento che \(\r\) è periodica, la funzione integranda di
\[
	\q(t) = \q_0 + \int_0^t \frac{L}{m\,\r^2(s)}\,\dd s
\]
sarà periodica di periodo \(T_0\).
Inoltre avremo
\[
	\w_1 \in \left( \frac{L}{m\,\r^2_-}, \frac{L}{m\,\r^2_+} \right) \qquad\text{e}\qquad \overline{\w_1} = \int_0^{T_0} \frac{\dd t}{T_0}\,\w_1(t).\graffito{\(\overline{\w_1}\) è il valor medio di \(\w_1\)}
\]
Da cui
\[
	\w_1(t) = \overline{\w_1} + f(t),
\]
dove \(f\) è una funzione periodica a media nulla di periodo \(T_0\). Segue
\[
	\q(t) = \q_0 + \overline{\w_1} t + \int_0^t f(s)\,\dd s,
\]
dove, nuovamente, l'integrale è una funzione periodica a media nulla di periodo \(T_0\).
Pertanto \(\q(t)\) corrisponde ad un moto circolare uniforme di periodo

\begin{remark}{Periodo del moto angolare}{periodoMotoAngolare}
	\[
		T_1 = \frac{2\p}{\frac{1}{T_0}\int_0^{T_0}\frac{L}{m\,\r^2(s)}\,\dd s}
	\]
\end{remark}
\noindent
Il moto complessivo è qualitativamente

FIGURA!!

Esso avviane pertanto nella corona circolare \(\r_-, \r_+\) con 
\[
	\Delta\q = \w_1 T_0 = \int_0^{T_0} \frac{L}{m\,\r^2(s)}\,\dd s.
\]

\begin{oss}
	Se \(\w_1 T_0\) è un multiplo razionale di \(2\p\), allora il moto complessivo è \emph{periodico} di periodo \(T_0 n = T_1 m\). Altrimenti si dice che è \emph{quasi periodico}.
\end{oss}